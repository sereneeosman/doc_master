%% Generated by Sphinx.
\def\sphinxdocclass{report}
\documentclass[letterpaper,10pt,english]{sphinxmanual}
\ifdefined\pdfpxdimen
   \let\sphinxpxdimen\pdfpxdimen\else\newdimen\sphinxpxdimen
\fi \sphinxpxdimen=.75bp\relax
\ifdefined\pdfimageresolution
    \pdfimageresolution= \numexpr \dimexpr1in\relax/\sphinxpxdimen\relax
\fi
%% let collapsible pdf bookmarks panel have high depth per default
\PassOptionsToPackage{bookmarksdepth=5}{hyperref}
%% turn off hyperref patch of \index as sphinx.xdy xindy module takes care of
%% suitable \hyperpage mark-up, working around hyperref-xindy incompatibility
\PassOptionsToPackage{hyperindex=false}{hyperref}
%% memoir class requires extra handling
\makeatletter\@ifclassloaded{memoir}
{\ifdefined\memhyperindexfalse\memhyperindexfalse\fi}{}\makeatother

\PassOptionsToPackage{booktabs}{sphinx}
\PassOptionsToPackage{colorrows}{sphinx}

\PassOptionsToPackage{warn}{textcomp}

\catcode`^^^^00a0\active\protected\def^^^^00a0{\leavevmode\nobreak\ }
\usepackage{cmap}
\usepackage{fontspec}
\defaultfontfeatures[\rmfamily,\sffamily,\ttfamily]{}
\usepackage{amsmath,amssymb,amstext}
\usepackage{polyglossia}
\setmainlanguage{english}



\setmainfont{FreeSerif}[
  Extension      = .otf,
  UprightFont    = *,
  ItalicFont     = *Italic,
  BoldFont       = *Bold,
  BoldItalicFont = *BoldItalic
]
\setsansfont{FreeSans}[
  Extension      = .otf,
  UprightFont    = *,
  ItalicFont     = *Oblique,
  BoldFont       = *Bold,
  BoldItalicFont = *BoldOblique,
]
\setmonofont{FreeMono}[
  Extension      = .otf,
  UprightFont    = *,
  ItalicFont     = *Oblique,
  BoldFont       = *Bold,
  BoldItalicFont = *BoldOblique,
]



\usepackage[Bjarne]{fncychap}
\usepackage{sphinx}

\fvset{fontsize=\small}
\usepackage{geometry}


% Include hyperref last.
\usepackage{hyperref}
% Fix anchor placement for figures with captions.
\usepackage{hypcap}% it must be loaded after hyperref.
% Set up styles of URL: it should be placed after hyperref.
\urlstyle{same}

\addto\captionsenglish{\renewcommand{\contentsname}{Overview:}}

\usepackage{sphinxmessages}
\setcounter{tocdepth}{0}


        \usepackage{fontspec}
        \setmainfont{DejaVu Serif}  % Use a font that supports your characters
    

\title{SereneeOsman Guidance}
\date{May 09, 2024}
\release{0.0.1}
\author{Serenee Osman}
\newcommand{\sphinxlogo}{\vbox{}}
\renewcommand{\releasename}{Release}
\makeindex
\begin{document}

\pagestyle{empty}
\sphinxmaketitle
\pagestyle{plain}
\sphinxtableofcontents
\pagestyle{normal}
\phantomsection\label{\detokenize{index::doc}}


\sphinxstepscope


\chapter{About Me}
\label{\detokenize{Other/About_Me:about-me}}\label{\detokenize{Other/About_Me::doc}}
\sphinxAtStartPar
I am Serenee Osman, originally from Sri Lanka.

\sphinxAtStartPar
I am graduated with a degree in Photogrammetry.
Since 2023, I have been a student at Ulster University. My journey into documentation began during my first\sphinxhyphen{}year programmin assignment.

\sphinxAtStartPar
Currently, I am employed as an LiDAR engineer for a Japanese company and residing in Japan.

\sphinxstepscope


\chapter{NI Tourist Map}
\label{\detokenize{egm722_serenee/NI_Tourist_Map_doc:ni-tourist-map}}\label{\detokenize{egm722_serenee/NI_Tourist_Map_doc::doc}}
\sphinxAtStartPar
\sphinxstylestrong{Explore Northern Ireland : Tourist Map with Integrated Transportation Hubs, and GP Surgeries.}

\sphinxAtStartPar
\sphinxstyleemphasis{link for repositories:}



\sphinxAtStartPar
\sphinxcode{\sphinxupquote{Download PDF}}


\section{Contents}
\label{\detokenize{egm722_serenee/NI_Tourist_Map_doc:contents}}\begin{itemize}
\item {} 
\sphinxAtStartPar
Overview

\item {} 
\sphinxAtStartPar
Data Provided

\item {} 
\sphinxAtStartPar
Setup

\item {} 
\sphinxAtStartPar
Getting Started

\end{itemize}


\section{Overview}
\label{\detokenize{egm722_serenee/NI_Tourist_Map_doc:overview}}
\sphinxAtStartPar
The “Explore Northern Ireland” script serves as a versatile and convenient tool tailored for anyone exploring Northern Ireland, be it tourists or residents.
This script creates an interactive map of comprehensive information on tourist sites, nearest transportation hubs (including both bus and train stations), as well as the distances between these transportation hubs and tourist sites. Moreover, it seamlessly integrates General Practitioner (GP) surgeries for emergency services, utilizing postal codes for easy searchability. By combining these features, the script enhances the overall travel experience, prioritizing safety and preparedness throughout every stage of the journey.




\subsection{Objectives:}
\label{\detokenize{egm722_serenee/NI_Tourist_Map_doc:objectives}}\begin{itemize}
\item {} 
\sphinxAtStartPar
\sphinxstylestrong{Interactive Map Creation}:Utilizing GeoDataFrame, the script generates an interactive map that visually presents tourist sites across Northern Ireland.

\item {} 
\sphinxAtStartPar
\sphinxstylestrong{Tourist Site Information}:Detailed information about each tourist site is included on the map, aiding users in making informed decisions about their destinations.

\item {} 
\sphinxAtStartPar
\sphinxstylestrong{Find nearest Transportation Hub}:The script identifies and displays the nearest transportation hubs, comprising both bus and train stations, in proximity to tourist sites, facilitating travel route planning and access to public transportation.

\item {} 
\sphinxAtStartPar
\sphinxstylestrong{Find nearest GP Surgeries}:General Practitioner (GP) surgeries are integrated into the map, allowing users to locate emergency medical services easily. Postal codes are leveraged for efficient searchability, ensuring prompt access to medical assistance.

\item {} 
\sphinxAtStartPar
\sphinxstylestrong{Distance Calculation}:Distances between transportation hubs/GP Surgeries and tourist sites are calculated and provided, enabling users to estimate travel times and plan itineraries effectively.

\end{itemize}


\subsection{Expected Results:}
\label{\detokenize{egm722_serenee/NI_Tourist_Map_doc:expected-results}}
\sphinxAtStartPar
Upon running the script, users will observe the following outcomes:

\sphinxAtStartPar
\sphinxstylestrong{Base Map with Counties and Outlines}: The function generates a base map of Northern Ireland featuring colored county boundaries overlaid with the country outline.
\sphinxstylestrong{Tourist Sites Overlay}: Tourist site polygons are overlaid onto the base map, facilitating visual identification of their locations. The color of each polygon corresponds to the county name, aiding in easy identification. Detailed information about each tourist site, such as its name, nearest transportation hubs, and distance, is conveniently accessible through pop\sphinxhyphen{}up windows. Moreover, the pop\sphinxhyphen{}ups specify the nearest GP surgery and its distance, cross\sphinxhyphen{}referenced with postal codes, enhancing user understanding.
\sphinxstylestrong{Coastal tourist Points Display}: Coastal spots data from a GeoJSON file is showcased on the map, potentially as markers. Pop\sphinxhyphen{}ups offer comprehensive details about each spot, including name, website URL, nearest transportation hubs, and distance. Moreover, the pop\sphinxhyphen{}ups highlight the nearest GP surgery, cross\sphinxhyphen{}referenced with postal codes, and its distance from each coastal spot.
\sphinxstylestrong{Interactive Features}: Additional data from a CSV/GeoJSON file enriches the map with attribute information about tourist sites, transportation hubs, and GP surgeries. Interactive elements such as pop\sphinxhyphen{}ups provide detailed insights when users hover over specific points of interest.
\sphinxstylestrong{Final Map Visualization}: Once all data is integrated and displayed, the function will return an interactive Folium map object that users can view in a web browser. This map will contain all the layers mentioned above, providing a comprehensive overview of tourist attractions, transportation hubs, GP surgeries,postal codes,distance values and coastal spots in Northern Ireland.
\sphinxstylestrong{Save as HTML}: The function may save the generated Folium map as an HTML file, allowing users to access it offline or embed it in web pages.


\section{Data Provided}
\label{\detokenize{egm722_serenee/NI_Tourist_Map_doc:data-provided}}
\sphinxAtStartPar
In the data\_file folder,contains are as follows:
\begin{itemize}
\item {} 
\sphinxAtStartPar
\sphinxcode{\sphinxupquote{NI\_Outline.shp}}, a shapefile comprising the Northern Ireland country outline.

\item {} 
\sphinxAtStartPar
\sphinxcode{\sphinxupquote{NI\_Counties.shp}}, a shapefile containing the boundaries of Northern Ireland’s counties

\item {} 
\sphinxAtStartPar
\sphinxcode{\sphinxupquote{NI\_Tourist\_Sites.shp}} , a shapefile containing polygon data of Historical Parks and Gardens.

\item {} 
\sphinxAtStartPar
\sphinxcode{\sphinxupquote{NI\_Coastal\_spots.geojson}}, a GeoJSON file containing information on Places to Visit in Causeway Coast and Glens.

\item {} 
\sphinxAtStartPar
\sphinxcode{\sphinxupquote{NI\_PostCodes\_GP.geojson}}, a GeoJSON file contain GP surgery information with postal codes in Northern Ireland.

\item {} 
\sphinxAtStartPar
\sphinxcode{\sphinxupquote{NI\_Tourist\_trans\_GP\_Dist.csv}},a csv file with information about nearest Transport Hub and Nearest GP surgery information.

\end{itemize}

\sphinxAtStartPar
The script  \sphinxstylestrong{Integrated\_Data\_Analysis.ipynb/ .py} outline the  process of \sphinxcode{\sphinxupquote{Re\sphinxhyphen{}Projection}}, \sphinxcode{\sphinxupquote{Polygon Clipping}}, and the creation of \sphinxcode{\sphinxupquote{NI\_Costal\_spots.geojson}} file, \sphinxcode{\sphinxupquote{NI\_Costal\_spots.geojson}}, \sphinxcode{\sphinxupquote{NI\_PostCodes\_GP.geojson}} and \sphinxcode{\sphinxupquote{NI\_Tourist\_trans\_GP\_Dist.csv}} files. To execute the script,  ensure you download the specified files into the \sphinxstylestrong{data\_files/download\_data} folder. Remember to extract shapefiles from \sphinxstylestrong{.zip} archives before use.
\begin{itemize}
\item {} 
\sphinxAtStartPar
\sphinxcode{\sphinxupquote{OSNI\_Open\_Data\_\sphinxhyphen{}\_Largescale\_Boundaries\_\sphinxhyphen{}\_NI\_Outline.shp}}, A shapefile comprising the Northern Ireland country outline was obtained from \sphinxhref{https://www.data.gov.uk/dataset/738c0cac-d330-4ba9-a2a5-8956383fb4a9/osni-open-data-largescale-boundaries-ni-outline}{Data.gov.uk} .

\item {} 
\sphinxAtStartPar
\sphinxcode{\sphinxupquote{OSNI\_Open\_Data\_\sphinxhyphen{}\_Largescale\_Boundaries\_\sphinxhyphen{}\_County\_Boundaries\_.shp}}, A shapefile containing the boundaries of Northern Ireland’s counties was sourced from \sphinxhref{https://admin.opendatani.gov.uk/dataset/osni-open-data-largescale-boundaries-county-boundaries}{OpenDataNI}.

\item {} 
\sphinxAtStartPar
\sphinxcode{\sphinxupquote{historic\sphinxhyphen{}parks\sphinxhyphen{}and\sphinxhyphen{}gardens.shp}} , A shapefile containing polygon data of Historic Parks and Gardens(valid as of April 2024) was obtained from \sphinxhref{https://admin.opendatani.gov.uk/dataset/historic-parks-and-gardens/resource/1f59b6a5-4f8d-4456-9009-e00586062b4d}{OpenDataNI}.

\item {} 
\sphinxAtStartPar
\sphinxcode{\sphinxupquote{Places\_to\_Visit\_in\_Causeway\_Coast\_and\_Glens.shp}}, A shapefile file containing information on Places to Visit in Causeway Coast and Glens sourced from \sphinxhref{https://admin.opendatani.gov.uk/dataset/places-to-visit-in-causeway-coast-and-glens}{OpenDataNI}.

\item {} 
\sphinxAtStartPar
\sphinxcode{\sphinxupquote{translink\sphinxhyphen{}stations\sphinxhyphen{}ni.geojson}}, a GeoJSON file containing the locations of all Bus and Rail stations in Northern Ireland, from \sphinxhref{https://www.opendatani.gov.uk/@translink/translink-ni-railways-stations}{OpenDataNI}.

\item {} 
\sphinxAtStartPar
\sphinxcode{\sphinxupquote{ukpostcodes.csv}}, a csv file contain all postcodes in united Kingdom from \sphinxhref{https://www.freemaptools.com/download-uk-postcode-lat-lng.htm\#google\_vignette}{FreeMapTools}.

\item {} 
\sphinxAtStartPar
\sphinxcode{\sphinxupquote{gp\sphinxhyphen{}practice\sphinxhyphen{}reference\sphinxhyphen{}file\sphinxhyphen{}\sphinxhyphen{}\sphinxhyphen{}jan\sphinxhyphen{}2024.csv}}, a csv file contain General Practitioner (GP) surgeries information (valid as of April 2024), from \sphinxhref{https://www.opendatani.gov.uk/@business-services-organisation/gp-practice-list-sizes}{OpenDataNI}.

\end{itemize}

\sphinxAtStartPar
\sphinxstylestrong{Important Note}: All datasets utilize the same Coordinate Reference System (\sphinxhref{https://geopandas.org/en/stable/docs/user\_guide/projections.html}{CRS}) , specifically the EPSG code for WGS84 latitude/longitude (\sphinxhref{https://epsg.io/4326}{EPSG:4326}). This consistency enables seamless integration of map data onto  \sphinxcode{\sphinxupquote{folium.map}}.


\section{Setup}
\label{\detokenize{egm722_serenee/NI_Tourist_Map_doc:setup}}

\subsection{1. Getting Started}
\label{\detokenize{egm722_serenee/NI_Tourist_Map_doc:getting-started}}
\sphinxAtStartPar
To begin the exercises, ensure you have both \sphinxcode{\sphinxupquote{git}} and \sphinxcode{\sphinxupquote{conda}} installed on your computer. Here’s a concise guide for installing Git \sphinxhref{https://docs.github.com/en/get-started/start-your-journey/creating-an-account-on-github}{Creating an account} , \sphinxhref{https://docs.github.com/en/desktop/installing-and-authenticating-to-github-desktop/setting-up-github-desktop}{GitHub Desktop} and \sphinxhref{https://docs.anaconda.com/free/anaconda/install/windows/}{Anaconda}.


\subsection{2. Download/clone repository}
\label{\detokenize{egm722_serenee/NI_Tourist_Map_doc:download-clone-repository}}
\sphinxAtStartPar
After installing Git and Anaconda, proceed to \sphinxstylestrong{clone} this repository to your computer using one of the following methods:
1. \sphinxstylestrong{Forking this Repository} : \sphinxhref{https://github.com/login}{Sign\sphinxhyphen{}in} to your created GitHub account and head to \sphinxhref{https://github.com/sereneeosman/egm722\_serenee}{sereneeosman/egm722\_serenee} repository. Click on the \sphinxcode{\sphinxupquote{Fork}} button located in the upper\sphinxhyphen{}right corner of the Window.This action duplicates the entire repository to your GitHub account.
2. \sphinxstylestrong{Cloning the repository} : Launch GitHub Desktop and navigate to \sphinxstylestrong{File} > \sphinxstylestrong{Clone Repository}. You’ll find your forked version of the \sphinxcode{\sphinxupquote{sereneeosman/egm722\_serenee}} repository repository listed under the \sphinxcode{\sphinxupquote{GitHub.com}} tab. Choose the repository and designate a local path where you want to save it (remember this path). Click on the “Clone” button. A new window will appear, showing the progress of downloading and unpacking files. Once completed, the repository will be set up on your local computer.
3. Another method to clone this repository is by clicking the green \sphinxstylestrong{“<> Code”} button on the GitHub repository page and selecting \sphinxstylestrong{“Download ZIP”} from the dropdown menu at the bottom. After downloading, unzip the file to your desired local path. Then, in GitHub Desktop, navigate to \sphinxstylestrong{File} > \sphinxstylestrong{Add Local Repository}. Although I don’t recommend this approach, it can be useful in certain cases.


\subsection{3. Create a conda Environment}
\label{\detokenize{egm722_serenee/NI_Tourist_Map_doc:create-a-conda-environment}}
\sphinxAtStartPar
Once you’ve successfully cloned the repository, you can proceed to create a \sphinxcode{\sphinxupquote{conda}} environment. To do this, utilize the provided \sphinxcode{\sphinxupquote{environment.yml}} file within the repository. If you’re using Anaconda Navigator, you can import the environment by selecting \sphinxstylestrong{“Import”} from the Environments panel and navigating to the \sphinxstylestrong{.yml} file in the local repository path.

\sphinxAtStartPar
If you prefer, you can open a command prompt (on Windows, navigate to the “Anaconda Prompt”). Then, go to the directory where you cloned this repository and execute the following command:

\begin{sphinxVerbatim}[commandchars=\\\{\}]
\PYG{n}{C}\PYG{p}{:}\PYGZbs{}\PYG{n}{Users}\PYGZbs{}\PYG{n}{sereneeosman}\PYGZbs{}\PYG{n}{egm722\PYGZus{}serenee}\PYG{o}{\PYGZgt{}} \PYG{n}{conda} \PYG{n}{env} \PYG{n}{create} \PYG{o}{\PYGZhy{}}\PYG{n}{f} \PYG{n}{environment}\PYG{o}{.}\PYG{n}{yml}
\end{sphinxVerbatim}

\sphinxAtStartPar
Setting up the \sphinxcode{\sphinxupquote{conda}} environment might take some time, but this process only needs to be done once per repository.


\subsection{4. Launch Jupyter Lab}
\label{\detokenize{egm722_serenee/NI_Tourist_Map_doc:launch-jupyter-lab}}
\sphinxAtStartPar
In Anaconda Navigator, you can launch Jupyter Lab and navigate to the local folder where this repository is located. Ensure that your \sphinxcode{\sphinxupquote{egm722\_serenee}} environment is activated.
Alternatively, from the command line, start by opening Anaconda Prompt and navigating to the folder where you’ve cloned the repository. Activate your newly\sphinxhyphen{}created environment with

\begin{sphinxVerbatim}[commandchars=\\\{\}]
\PYG{n}{conda} \PYG{n}{activate} \PYG{n}{egm722\PYGZus{}serenee}
\end{sphinxVerbatim}

\sphinxAtStartPar
Then, execute the command

\begin{sphinxVerbatim}[commandchars=\\\{\}]
\PYG{n}{jupyter}\PYG{o}{\PYGZhy{}}\PYG{n}{lab}
\end{sphinxVerbatim}

\sphinxAtStartPar
This action should open a web browser window, providing an overview of the current folder.


\subsection{\#\# 5. Repository Structure}
\label{\detokenize{egm722_serenee/NI_Tourist_Map_doc:repository-structure}}\begin{itemize}
\item {} 
\sphinxAtStartPar
\sphinxcode{\sphinxupquote{NI\_TouristMap.ipynb}} :This file contains the main code for creating a tourist map. It serves as the primary navigation point for executing the code related to the creation of the tourist map.

\item {} 
\sphinxAtStartPar
\sphinxcode{\sphinxupquote{Integrated\_Data\_Analysis.ipynb/.py}} :This file demonstrates how to integrate downloaded data and perform analysis on it. It provides insights into the process of combining different datasets and conducting analysis tasks, available both in Jupyter Notebook (.ipynb) and Python script (.py) formats.

\end{itemize}


\section{Getting Started}
\label{\detokenize{egm722_serenee/NI_Tourist_Map_doc:id5}}
\sphinxAtStartPar
To get started, open Jupyter Notebook and begin working through the notebook titled “NI\_TouristMap\_edit.ipynb”.
To execute the cell, highlight it by clicking on it, then either press \sphinxstylestrong{Ctrl + Enter} or click the triangular \sphinxstylestrong{play} button located at the top of this panel.

\sphinxAtStartPar
\sphinxstylestrong{Importing Libraries}

\sphinxAtStartPar
To get started, first import the required python libraries.
* \sphinxcode{\sphinxupquote{os}}: This library provides a way to interact with the operating system, such as managing files and directories.(\sphinxhref{https://docs.python.org/3/library/os.html}{Documentation})
* \sphinxcode{\sphinxupquote{pandas}} (aliased as \sphinxstylestrong{pd}): A powerful data manipulation library that allows you to work with structured data (e.g., data frames,Comma Separated Value (CSV) file).(\sphinxhref{https://pandas.pydata.org/}{Documentation})
* \sphinxcode{\sphinxupquote{geopandas}} (aliased as \sphinxstylestrong{gpd}): An extension of Pandas specifically designed for working with geospatial data (e.g., vector data).(\sphinxhref{https://geopandas.org/en/stable/}{Documentation})
* \sphinxcode{\sphinxupquote{folium}}: A Python library for creating interactive maps.(\sphinxhref{https://python-visualization.github.io/folium/latest/}{Documentation})

\begin{sphinxVerbatim}[commandchars=\\\{\}]
\PYG{k+kn}{import} \PYG{n+nn}{os}
\PYG{k+kn}{import} \PYG{n+nn}{pandas} \PYG{k}{as} \PYG{n+nn}{pd}
\PYG{k+kn}{import} \PYG{n+nn}{geopandas} \PYG{k}{as} \PYG{n+nn}{gpd}
\PYG{k+kn}{import} \PYG{n+nn}{folium}
\end{sphinxVerbatim}

\sphinxAtStartPar
\sphinxstylestrong{Reading Geospatial Data}

\sphinxAtStartPar
The code utilizes GeoPandas’ \sphinxcode{\sphinxupquote{.read\_file()}} function (\sphinxhref{https://geopandas.org/en/stable/docs/reference/api/geopandas.read\_file.html}{Documentation}) to read geospatial data from shapefiles. When handling shapefiles, they are treated as \sphinxstylestrong{GeoDataFrame} (\sphinxhref{https://geopandas.org/en/stable/docs/reference/api/geopandas.GeoDataFrame.html}{Documentation} ), resembling attribute tables but with added geospatial functionalities.

\sphinxAtStartPar
A \sphinxstylestrong{GeoDataFrame} enhances a Pandas \sphinxstylestrong{DataFrame} (\sphinxhref{https://pandas.pydata.org/docs/reference/api/pandas.DataFrame.html}{Documentation}) by integrating geospatial capabilities.
It stored geometry for each feature \sphinxstyleemphasis{points}, \sphinxstyleemphasis{lines}, \sphinxstyleemphasis{polygons}, along with associated attributes. With GeoDataFrames, users can conduct spatial operations and effectively visualize data on maps.

\sphinxAtStartPar
Initially, we’ll read data for country outlines and counties to establish the foundation of our map.

\begin{sphinxVerbatim}[commandchars=\\\{\}]
\PYG{c+c1}{\PYGZsh{} Read the shapefiles}
\PYG{n}{outline} \PYG{o}{=} \PYG{n}{gpd}\PYG{o}{.}\PYG{n}{read\PYGZus{}file}\PYG{p}{(}\PYG{n}{os}\PYG{o}{.}\PYG{n}{path}\PYG{o}{.}\PYG{n}{abspath}\PYG{p}{(}\PYG{l+s+s2}{\PYGZdq{}}\PYG{l+s+s2}{data\PYGZus{}files/NI\PYGZus{}Outline.shp}\PYG{l+s+s2}{\PYGZdq{}}\PYG{p}{)}\PYG{p}{)} \PYG{c+c1}{\PYGZsh{} Path to the input shapefile of Country Outline data}
\PYG{n}{counties} \PYG{o}{=} \PYG{n}{gpd}\PYG{o}{.}\PYG{n}{read\PYGZus{}file}\PYG{p}{(}\PYG{n}{os}\PYG{o}{.}\PYG{n}{path}\PYG{o}{.}\PYG{n}{abspath}\PYG{p}{(}\PYG{l+s+s2}{\PYGZdq{}}\PYG{l+s+s2}{data\PYGZus{}files/NI\PYGZus{}Counties.shp}\PYG{l+s+s2}{\PYGZdq{}}\PYG{p}{)}\PYG{p}{)} \PYG{c+c1}{\PYGZsh{} Path to the input shapefile of Counties data}
\end{sphinxVerbatim}

\sphinxAtStartPar
\sphinxstylestrong{Create a Base map}

\sphinxAtStartPar
To generate an interactive map from \sphinxstylestrong{GeoDataFrames}, we utilize the \sphinxcode{\sphinxupquote{.explore}} (\sphinxhref{https://geopandas.org/en/stable/docs/reference/api/geopandas.GeoDataFrame.explore.html}{Documentation}) function  which generates a \sphinxstylestrong{folium.Map} (\sphinxhref{https://geopandas.org/en/stable/docs/user\_guide/interactive\_mapping.html}{Documentation}).
We assign the result to \sphinxcode{\sphinxupquote{m = folium.Map}} (\sphinxhref{https://python-visualization.github.io/folium/latest/getting\_started.html\#Creating-a-map}{Documentation}), creating a base map.

\sphinxAtStartPar
We will utilize the “CountiesName” column to visualize the each polygon, and apply the \sphinxstylestrong{Set2} colormap from \sphinxcode{\sphinxupquote{matplotlib}} to set the colors.More information about colormaps can be found \sphinxhref{https://matplotlib.org/stable/users/explain/colors/colormaps.html}{here}.
In this case, each county will be assigned a color based on its name.

\begin{sphinxVerbatim}[commandchars=\\\{\}]
\PYG{c+c1}{\PYGZsh{} Create a Base Map on Counties name.}
\PYG{n}{m} \PYG{o}{=} \PYG{n}{counties}\PYG{o}{.}\PYG{n}{explore}\PYG{p}{(}\PYG{l+s+s2}{\PYGZdq{}}\PYG{l+s+s2}{CountyName}\PYG{l+s+s2}{\PYGZdq{}}\PYG{p}{,} \PYG{n}{cmap} \PYG{o}{=} \PYG{l+s+s2}{\PYGZdq{}}\PYG{l+s+s2}{Set2}\PYG{l+s+s2}{\PYGZdq{}}\PYG{p}{)}
\end{sphinxVerbatim}

\sphinxAtStartPar
\sphinxstylestrong{Adding Country outline into Base folium map}

\sphinxAtStartPar
Next we will add country outline into the base map from read shapefile (\sphinxcode{\sphinxupquote{NI\_Outline.shp}}).

\sphinxAtStartPar
To do this we use \sphinxcode{\sphinxupquote{folium.GeoJson()}} (\sphinxhref{https://python-visualization.github.io/folium/latest/user\_guide/geojson/geojson.html}{Documentation}) function in the folium library.

\sphinxAtStartPar
GeoJSON is a format for encoding a variety of geographic data structures,JSON (JavaScript Object Notation) format. It’s commonly used in web mapping applications and spatial databases to represent geographic features such as \sphinxstyleemphasis{points}, \sphinxstyleemphasis{lines}, \sphinxstyleemphasis{polygons}, or a set of coordinates.
This data defines the shape of an area on the map.
\begin{itemize}
\item {} 
\sphinxAtStartPar
\sphinxcode{\sphinxupquote{*outline}}: This is the GeoDataFrame containing the outline shape data.

\item {} 
\sphinxAtStartPar
\sphinxcode{\sphinxupquote{style\_function=lambda feature}} (\sphinxhref{https://python-visualization.github.io/folium/latest/user\_guide/geojson/geojson.html\#Styling}{Documentation})  : This is an argument passed to customize the style of the GeoJSON features. It’s a \sphinxcode{\sphinxupquote{lambda function}} that takes a feature as input and returns a dictionary specifying the style properties.

\item {} 
\sphinxAtStartPar
\sphinxcode{\sphinxupquote{"color": "black"}},  : This sets the color of the outline to black.

\item {} 
\sphinxAtStartPar
\sphinxcode{\sphinxupquote{"fillOpacity": 0}} : This sets the fill opacity of the outline to 0, meaning it will be transparent and won’t fill the area inside the outline.

\item {} 
\sphinxAtStartPar
\sphinxcode{\sphinxupquote{name="outline"}}  : This sets the name of the GeoJson layer to ‘outline’. This name can be used to control the visibility of the layer in the folium map’s layer control.

\item {} 
\sphinxAtStartPar
\sphinxcode{\sphinxupquote{.add\_to(m)}} : This method adds the GeoJson layer to the base folium map (m).

\end{itemize}

\begin{sphinxVerbatim}[commandchars=\\\{\}]
\PYG{c+c1}{\PYGZsh{} Add the outline with a black frame}
\PYG{n}{folium}\PYG{o}{.}\PYG{n}{GeoJson}\PYG{p}{(}
    \PYG{n}{outline}\PYG{p}{,} \PYG{c+c1}{\PYGZsh{} outline shape data}
    \PYG{n}{style\PYGZus{}function}\PYG{o}{=}\PYG{k}{lambda} \PYG{n}{feature}\PYG{p}{:} \PYG{p}{\PYGZob{}}  \PYG{c+c1}{\PYGZsh{} customize the style of the GeoJSON features}
        \PYG{l+s+s2}{\PYGZdq{}}\PYG{l+s+s2}{color}\PYG{l+s+s2}{\PYGZdq{}}\PYG{p}{:} \PYG{l+s+s2}{\PYGZdq{}}\PYG{l+s+s2}{black}\PYG{l+s+s2}{\PYGZdq{}}\PYG{p}{,}  \PYG{c+c1}{\PYGZsh{} sets the color of the outline to black}
        \PYG{l+s+s2}{\PYGZdq{}}\PYG{l+s+s2}{fillOpacity}\PYG{l+s+s2}{\PYGZdq{}}\PYG{p}{:} \PYG{l+m+mi}{0} \PYG{c+c1}{\PYGZsh{} sets the fill opacity as transparent}
    \PYG{p}{\PYGZcb{}}\PYG{p}{,}
    \PYG{n}{name}\PYG{o}{=}\PYG{l+s+s2}{\PYGZdq{}}\PYG{l+s+s2}{outline}\PYG{l+s+s2}{\PYGZdq{}} \PYG{c+c1}{\PYGZsh{}name of the GeoJson layer}
\PYG{p}{)}\PYG{o}{.}\PYG{n}{add\PYGZus{}to}\PYG{p}{(}\PYG{n}{m}\PYG{p}{)} \PYG{c+c1}{\PYGZsh{}adds the GeoJson layer to the base folium map **m**.}
\end{sphinxVerbatim}

\sphinxAtStartPar
If you encounter the result \sphinxstylestrong{<folium.features.GeoJson at 0x177bc95bb60>} without an error message, it signifies that the GeoJSON layer object has been successfully created and added to the map (m).

\sphinxAtStartPar
Display the base folium map

\begin{sphinxVerbatim}[commandchars=\\\{\}]
\PYG{n}{m}
\end{sphinxVerbatim}

\sphinxAtStartPar
As depicted, a color legend is incorporated at the bottom right\sphinxhyphen{}hand corner of the map, providing information on the colors assigned to each polygon. Additionally, a scale is situated at the bottom left\sphinxhyphen{}hand corner of the map. The country outline of Northern Ireland is displayed with black border lines.You can zoom in or out to examine finer details, including those on the {[}OpenStreetMap{]}(\sphinxurl{https://www.openstreetmap.org/\#map=5/35.588/134.380}) base layer.

\sphinxAtStartPar
\sphinxstylestrong{Convert DataFrame to GeoDataFrame}

\sphinxAtStartPar
Convert csv data to vector data

\sphinxAtStartPar
The \sphinxcode{\sphinxupquote{pd.read\_csv()}} (\sphinxhref{https://pandas.pydata.org/pandas-docs/stable/reference/api/pandas.read\_csv.html}{Documentation}) function is used to read data from a CSV (Comma\sphinxhyphen{}Separated Values) file into a Pandas \sphinxstylestrong{DataFrame}. The function reads the contents of the CSV file and creates a DataFrame with the data.

\sphinxAtStartPar
\sphinxstylestrong{DataFrame} (\sphinxhref{https://pandas.pydata.org/docs/reference/api/pandas.DataFrame.html}{Documentation}) is a two\sphinxhyphen{}dimensional labeled data structure in Pandas. It organizes data into rows and columns, similar to a table. Each column in the DataFrame corresponds to a variable, and each row represents an observation.

\sphinxAtStartPar
We’ll read the integrated CSV file containing tourist site names, details of the nearest transport hubs, and details of the nearest GP surgeries. Refer to the \sphinxstylestrong{Integrated\_Data\_Analysis.ipynb} file for instructions on how to create this csv file.

\begin{sphinxVerbatim}[commandchars=\\\{\}]
\PYG{c+c1}{\PYGZsh{}read integrated csv file}
\PYG{n}{df} \PYG{o}{=} \PYG{n}{pd}\PYG{o}{.}\PYG{n}{read\PYGZus{}csv}\PYG{p}{(}\PYG{l+s+s2}{\PYGZdq{}}\PYG{l+s+s2}{data\PYGZus{}files/NI\PYGZus{}Tourist\PYGZus{}trans\PYGZus{}GP\PYGZus{}Dist.csv}\PYG{l+s+s2}{\PYGZdq{}}\PYG{p}{)}
\end{sphinxVerbatim}

\sphinxAtStartPar
The \sphinxcode{\sphinxupquote{.head()}} (\sphinxhref{https://pandas.pydata.org/docs/reference/api/pandas.DataFrame.head.html}{Documentation}) function is used to display the first few rows of the {\color{red}\bfseries{}``}df``(a \sphinxstylestrong{DataFrame}). By default, it returns the first five rows, but you can specify the number of rows you want to display by passing an integer argument to the function (e.g., df.head(10) would display the first ten rows)

\begin{sphinxVerbatim}[commandchars=\\\{\}]
\PYG{c+c1}{\PYGZsh{} Check the first few rows of df}
\PYG{n}{df}\PYG{o}{.}\PYG{n}{head}\PYG{p}{(}\PYG{p}{)}
\end{sphinxVerbatim}

\sphinxAtStartPar
Next, we’ll open the base shapefile of “tourist sites,” which includes polygon data.
The \sphinxcode{\sphinxupquote{gpd.read\_file}} (\sphinxhref{https://geopandas.org/en/stable/docs/reference/api/geopandas.read\_file.html}{Documentation}) function, reads the shapefile and returns a \sphinxstylestrong{GeoDataFrame}.

\begin{sphinxVerbatim}[commandchars=\\\{\}]
\PYG{c+c1}{\PYGZsh{} read tourist site polygon data}
\PYG{n}{tourist} \PYG{o}{=} \PYG{n}{gpd}\PYG{o}{.}\PYG{n}{read\PYGZus{}file}\PYG{p}{(}\PYG{n}{os}\PYG{o}{.}\PYG{n}{path}\PYG{o}{.}\PYG{n}{abspath}\PYG{p}{(}\PYG{l+s+s2}{\PYGZdq{}}\PYG{l+s+s2}{data\PYGZus{}files/NI\PYGZus{}Tourist\PYGZus{}Sites.shp}\PYG{l+s+s2}{\PYGZdq{}}\PYG{p}{)}\PYG{p}{)} \PYG{c+c1}{\PYGZsh{} path to the tourist site shapefile data}
\end{sphinxVerbatim}

\sphinxAtStartPar
The {\color{red}\bfseries{}``}.columns``(\sphinxhref{https://www.geeksforgeeks.org/python-pandas-dataframe-columns/}{Documents}) function display name of the field head in attributes of the DataFrame object in pandas.

\sphinxAtStartPar
When you input \sphinxcode{\sphinxupquote{tourist.columns}}, it returns an index of column labels within the “tourist” \sphinxstylestrong{DataFrame}.

\begin{sphinxVerbatim}[commandchars=\\\{\}]
\PYG{c+c1}{\PYGZsh{} Displaying the column names of the shapefile.}
\PYG{n}{tourist}\PYG{o}{.}\PYG{n}{columns}
\end{sphinxVerbatim}

\sphinxAtStartPar
This code merges a GeoDataFrame and a DataFrame, namely \sphinxstylestrong{tourist} and \sphinxstylestrong{df}, based on a common column in each dataset.
\begin{itemize}
\item {} 
\sphinxAtStartPar
The {\color{red}\bfseries{}``}.merge``(\sphinxhref{https://pandas.pydata.org/docs/reference/api/pandas.merge.html}{Documentation}) function provided attempts to merge the two datasets. Merge these datasets based on a common field (the “SITE”{[}tourist{]} and “Tourist Sites”{[}df{]} columns)

\item {} 
\sphinxAtStartPar
The \sphinxcode{\sphinxupquote{left\_on}} parameter specifies the column name in the left dataset (\sphinxstylestrong{tourist\_site}) to use for merging (in this case, “SITE”).

\item {} 
\sphinxAtStartPar
The \sphinxcode{\sphinxupquote{right\_on}} parameter specifies the column name in the right dataset (\sphinxstylestrong{df}) to use for merging (in this case, “Tourist Sites”).

\item {} 
\sphinxAtStartPar
The resulting \sphinxcode{\sphinxupquote{merge\_site}} DataFrame will contain combined rows from both datasets.

\item {} 
\sphinxAtStartPar
The \sphinxcode{\sphinxupquote{.head()}} method is then called on the merged DataFrame to display the first few rows.By default, it shows the first five rows, along with the column names.

\end{itemize}

\begin{sphinxVerbatim}[commandchars=\\\{\}]
\PYG{c+c1}{\PYGZsh{}Merge the CSV data (DataFrame) with the shapefile data (GeoDataFrame) based on a common column.}
\PYG{n}{merge\PYGZus{}site} \PYG{o}{=} \PYG{n}{tourist}\PYG{o}{.}\PYG{n}{merge}\PYG{p}{(}\PYG{n}{df}\PYG{p}{,} \PYG{n}{left\PYGZus{}on}\PYG{o}{=}\PYG{l+s+s2}{\PYGZdq{}}\PYG{l+s+s2}{SITE}\PYG{l+s+s2}{\PYGZdq{}}\PYG{p}{,} \PYG{n}{right\PYGZus{}on}\PYG{o}{=} \PYG{l+s+s2}{\PYGZdq{}}\PYG{l+s+s2}{Tourist Sites}\PYG{l+s+s2}{\PYGZdq{}}\PYG{p}{)}
\PYG{n}{merge\PYGZus{}site}\PYG{o}{.}\PYG{n}{head}\PYG{p}{(}\PYG{p}{)}
\end{sphinxVerbatim}

\sphinxAtStartPar
This code creates a new GeoDataFrame named “visit\_geo” by selecting specific columns from the previously merged DataFrame “merge\_site”.
\begin{itemize}
\item {} 
\sphinxAtStartPar
The first line of code selects specific columns from the merge\_site DataFrame.  Selection of the \sphinxstylestrong{geometry} column is important as it contain the coordinates and feature types. The geometry column is necessary for the second command line to generate the GeoDataFrame.

\item {} 
\sphinxAtStartPar
The second command \sphinxcode{\sphinxupquote{.GeoDataFrame}} (\sphinxhref{https://geopandas.org/en/stable/docs/reference/api/geopandas.GeoDataFrame.html}{Documentation})   converts the \sphinxstylestrong{DataFrame} named “visit\_filter” into a \sphinxstylestrong{GeoDataFrame}.

\item {} 
\sphinxAtStartPar
The \sphinxcode{\sphinxupquote{.head()}} function displays the first few rows  of the “visit\_geo” GeoDataFrame.

\end{itemize}

\begin{sphinxVerbatim}[commandchars=\\\{\}]
\PYG{n}{visit\PYGZus{}filter} \PYG{o}{=} \PYG{n}{merge\PYGZus{}site}\PYG{p}{[}\PYG{p}{[}\PYG{l+s+s2}{\PYGZdq{}}\PYG{l+s+s2}{Tourist Sites}\PYG{l+s+s2}{\PYGZdq{}}\PYG{p}{,} \PYG{l+s+s2}{\PYGZdq{}}\PYG{l+s+s2}{Near\PYGZus{}T\PYGZus{}Hub}\PYG{l+s+s2}{\PYGZdq{}}\PYG{p}{,}\PYG{l+s+s2}{\PYGZdq{}}\PYG{l+s+s2}{Trans\PYGZus{}Dist}\PYG{l+s+s2}{\PYGZdq{}}\PYG{p}{,}\PYG{l+s+s2}{\PYGZdq{}}\PYG{l+s+s2}{Near\PYGZus{}GP}\PYG{l+s+s2}{\PYGZdq{}}\PYG{p}{,} \PYG{l+s+s2}{\PYGZdq{}}\PYG{l+s+s2}{GP\PYGZus{}Dist}\PYG{l+s+s2}{\PYGZdq{}}\PYG{p}{,}\PYG{l+s+s2}{\PYGZdq{}}\PYG{l+s+s2}{PostCode}\PYG{l+s+s2}{\PYGZdq{}}\PYG{p}{,}\PYG{l+s+s2}{\PYGZdq{}}\PYG{l+s+s2}{geometry}\PYG{l+s+s2}{\PYGZdq{}}\PYG{p}{]}\PYG{p}{]}
\PYG{n}{visit\PYGZus{}geo} \PYG{o}{=} \PYG{n}{gpd}\PYG{o}{.}\PYG{n}{GeoDataFrame}\PYG{p}{(}\PYG{n}{visit\PYGZus{}filter}\PYG{p}{)}
\PYG{n}{visit\PYGZus{}geo}\PYG{o}{.}\PYG{n}{head}\PYG{p}{(}\PYG{p}{)}
\end{sphinxVerbatim}

\sphinxAtStartPar
We will add county names into the “visit\_geo” file, where geometries intersect.

\sphinxAtStartPar
The \sphinxcode{\sphinxupquote{.sjoin}} (\sphinxhref{https://geopandas.org/en/stable/docs/reference/api/geopandas.sjoin.html}{Documentation}) function allows spatial join of the two GeoDataFrames (\sphinxcode{\sphinxupquote{gpd.sjoin(left\_df, right\_df, how='inner')}}).
\begin{itemize}
\item {} 
\sphinxAtStartPar
\sphinxcode{\sphinxupquote{left\_df}}: The left GeoDataFrame (in this case, “visit\_geo”)

\item {} 
\sphinxAtStartPar
\sphinxcode{\sphinxupquote{right\_df}}: The right GeoDataFrame (in this case, “counties”)

\item {} 
\sphinxAtStartPar
\sphinxcode{\sphinxupquote{how:'inner'}}: Retains only the rows where geometries intersect in both GeoDataFrames.

\end{itemize}

\begin{sphinxVerbatim}[commandchars=\\\{\}]
\PYG{n}{visit\PYGZus{}merge} \PYG{o}{=} \PYG{n}{gpd}\PYG{o}{.}\PYG{n}{sjoin}\PYG{p}{(}\PYG{n}{visit\PYGZus{}geo}\PYG{p}{,}\PYG{n}{counties}\PYG{p}{,}\PYG{n}{how}\PYG{o}{=}\PYG{l+s+s2}{\PYGZdq{}}\PYG{l+s+s2}{inner}\PYG{l+s+s2}{\PYGZdq{}}\PYG{p}{)}
\end{sphinxVerbatim}

\sphinxAtStartPar
To verify the results \sphinxcode{\sphinxupquote{.head()}} function used to retrieve the first few rows (usually the top 5 rows) GeoDataFrame of “visit\_merge”. It provides a quick preview of the data contained within the GeoDataFrame.

\begin{sphinxVerbatim}[commandchars=\\\{\}]
\PYG{n}{visit\PYGZus{}merge}\PYG{o}{.}\PYG{n}{head}\PYG{p}{(}\PYG{p}{)}
\end{sphinxVerbatim}

\sphinxAtStartPar
You will see result GeoDatFrame contain both columns of the “Counties” file and “visit\_geo” file.

\sphinxAtStartPar
The next code filters specific columns from the \sphinxstylestrong{GeoDataFrame}, constructs a new GeoDataFrame from the filtered data.
This process is commonly used to focus on relevant columns and convert tabular data with geometric information into a format suitable for spatial analysis.

\sphinxAtStartPar
Then displays the first few rows of the resulting GeoDataFrame (“visit\_all”).

\begin{sphinxVerbatim}[commandchars=\\\{\}]
\PYG{n}{visit\PYGZus{}all} \PYG{o}{=} \PYG{n}{visit\PYGZus{}merge}\PYG{p}{[}\PYG{p}{[}\PYG{l+s+s2}{\PYGZdq{}}\PYG{l+s+s2}{Tourist Sites}\PYG{l+s+s2}{\PYGZdq{}}\PYG{p}{,} \PYG{l+s+s2}{\PYGZdq{}}\PYG{l+s+s2}{Near\PYGZus{}T\PYGZus{}Hub}\PYG{l+s+s2}{\PYGZdq{}}\PYG{p}{,}\PYG{l+s+s2}{\PYGZdq{}}\PYG{l+s+s2}{Trans\PYGZus{}Dist}\PYG{l+s+s2}{\PYGZdq{}}\PYG{p}{,}\PYG{l+s+s2}{\PYGZdq{}}\PYG{l+s+s2}{Near\PYGZus{}GP}\PYG{l+s+s2}{\PYGZdq{}}\PYG{p}{,} \PYG{l+s+s2}{\PYGZdq{}}\PYG{l+s+s2}{GP\PYGZus{}Dist}\PYG{l+s+s2}{\PYGZdq{}}\PYG{p}{,}\PYG{l+s+s2}{\PYGZdq{}}\PYG{l+s+s2}{PostCode}\PYG{l+s+s2}{\PYGZdq{}}\PYG{p}{,}\PYG{l+s+s2}{\PYGZdq{}}\PYG{l+s+s2}{geometry}\PYG{l+s+s2}{\PYGZdq{}}\PYG{p}{,}\PYG{l+s+s2}{\PYGZdq{}}\PYG{l+s+s2}{CountyName}\PYG{l+s+s2}{\PYGZdq{}}\PYG{p}{]}\PYG{p}{]}
\PYG{n}{visit\PYGZus{}all}\PYG{o}{.}\PYG{n}{head}\PYG{p}{(}\PYG{p}{)}
\end{sphinxVerbatim}

\sphinxAtStartPar
Next we will display the GeoDataFrame on the folium map and popup the attribute information.

\sphinxAtStartPar
The \sphinxcode{\sphinxupquote{.explore}} (\sphinxhref{https://geopandas.org/en/stable/docs/reference/api/geopandas.GeoDataFrame.explore.html}{Documents}) function is used to visualize the polygons of tourist site (Named “visit\_all”) base on the County name.This implies that the symbology is categorized according to the \sphinxstylestrong{county name}, assigning a single color to each polygon belonging to a specific county.
\begin{itemize}
\item {} 
\sphinxAtStartPar
\sphinxcode{\sphinxupquote{"CountyName"}}: Specifies the column to be visualized.

\item {} 
\sphinxAtStartPar
\sphinxcode{\sphinxupquote{cmap = "gist\_rainbow"}} : Assigning corresponding colors to each tourist site polygons base on the county name using \sphinxcode{\sphinxupquote{matplotlib}} colormap library.The more about \sphinxcode{\sphinxupquote{matplotlib}} library, defined “color map” (\sphinxhref{https://matplotlib.org/stable/users/explain/colors/colormaps.html}{Documents}).

\item {} 
\sphinxAtStartPar
\sphinxcode{\sphinxupquote{m=m}}: Sets the base map m to be displayed. If \sphinxcode{\sphinxupquote{m=None}}, it prevents recursion errors.

\item {} 
\sphinxAtStartPar
\sphinxcode{\sphinxupquote{popup=True}}: Enables popups to display additional information when interacting with the map.

\item {} 
\sphinxAtStartPar
\sphinxcode{\sphinxupquote{legend=False}}: Disables the display of the legend on the map.

\end{itemize}

\begin{sphinxVerbatim}[commandchars=\\\{\}]
\PYG{n}{visit\PYGZus{}all}\PYG{o}{.}\PYG{n}{explore}\PYG{p}{(}\PYG{l+s+s2}{\PYGZdq{}}\PYG{l+s+s2}{CountyName}\PYG{l+s+s2}{\PYGZdq{}}\PYG{p}{,} \PYG{c+c1}{\PYGZsh{} show the CountyName column}
                \PYG{n}{cmap}\PYG{o}{=}\PYG{l+s+s2}{\PYGZdq{}}\PYG{l+s+s2}{gist\PYGZus{}rainbow}\PYG{l+s+s2}{\PYGZdq{}}\PYG{p}{,} \PYG{c+c1}{\PYGZsh{} use the \PYGZdq{}hsv\PYGZdq{} colormap from matplotlib}
                \PYG{n}{m}\PYG{o}{=}\PYG{n}{m}\PYG{p}{,} \PYG{c+c1}{\PYGZsh{} set the base folium.map}
                \PYG{n}{popup} \PYG{o}{=} \PYG{k+kc}{True}\PYG{p}{,} \PYG{c+c1}{\PYGZsh{}Show information as popup when curser move on to the polygon}
                \PYG{n}{legend} \PYG{o}{=} \PYG{k+kc}{False}\PYG{p}{,} \PYG{c+c1}{\PYGZsh{}Don`t display a separated legend.}
\PYG{p}{)}
\end{sphinxVerbatim}

\sphinxAtStartPar
Adding Coastline visit spots into Folium map

\sphinxAtStartPar
The code reads a GeoJSON file named “NI\_Coastal\_spots.geojson” using the GeoPandas \sphinxcode{\sphinxupquote{read\_file}} (\sphinxhref{https://geopandas.org/en/stable/docs/reference/api/geopandas.read\_file.html}{Documents})function.

\sphinxAtStartPar
This GeoJSON file contains data about Places to Visit in Causeway Coast and Glens, including information about the nearest transport hub and its distance, nearest GP surgery details and distance, and additional data such as the URL for the website.
Refer to the \sphinxstylestrong{Integrated\_Data\_Analysis.ipynb} file for instructions on how to create the GeoJSON file.

\begin{sphinxVerbatim}[commandchars=\\\{\}]
\PYG{c+c1}{\PYGZsh{} read geojason file}
\PYG{n}{coastalpt} \PYG{o}{=} \PYG{n}{gpd}\PYG{o}{.}\PYG{n}{read\PYGZus{}file}\PYG{p}{(}\PYG{n}{os}\PYG{o}{.}\PYG{n}{path}\PYG{o}{.}\PYG{n}{abspath}\PYG{p}{(}\PYG{l+s+s2}{\PYGZdq{}}\PYG{l+s+s2}{data\PYGZus{}files/NI\PYGZus{}Coastal\PYGZus{}spots.geojson}\PYG{l+s+s2}{\PYGZdq{}}\PYG{p}{)}\PYG{p}{)}
\end{sphinxVerbatim}

\sphinxAtStartPar
\# Printing \sphinxcode{\sphinxupquote{coastline.head()}} would display the first few rows of the GeoDataFrame “coastalpt”.

\begin{sphinxVerbatim}[commandchars=\\\{\}]
\PYG{n}{coastalpt}\PYG{o}{.}\PYG{n}{head}\PYG{p}{(}\PYG{p}{)}
\end{sphinxVerbatim}

\sphinxAtStartPar
This code snippet defines a dictionary named coastline\_args containing parameters for configuring the display of “coastalpt” \sphinxcode{\sphinxupquote{markers}} (\sphinxhref{https://python-visualization.github.io/folium/latest/getting\_started.html\#Adding-markers}{Documents}) on a folium map.
\begin{itemize}
\item {} 
\sphinxAtStartPar
\sphinxcode{\sphinxupquote{"m": m}}: This parameter specifies the folium map (m) on which the coastline markers will be plotted. The value associated with this key is an existing folium map instance (m).

\item {} 
\sphinxAtStartPar
\sphinxcode{\sphinxupquote{"marker\_type": "marker"}}: This parameter specifies the type of marker to be used for the coastline. In this case, it’s set to “marker”, indicating standard point markers.

\item {} 
\sphinxAtStartPar
\sphinxcode{\sphinxupquote{"popup": True}}: This parameter determines whether popups will be displayed when clicking on the markers. By setting it to True, popups will be enabled, allowing additional information to be shown when interacting with the markers.

\item {} 
\sphinxAtStartPar
\sphinxcode{\sphinxupquote{"legend"}}: False: This parameter controls the display of a legend. Here, it’s set to False, indicating that no legend will be shown for the markers.

\item {} 
\sphinxAtStartPar
\sphinxcode{\sphinxupquote{"marker\_kwds": \{...\}}}: This parameter provides additional keyword arguments for styling the markers. Arguments based on the \sphinxcode{\sphinxupquote{folium.Map.Icon}} (\sphinxhref{https://python-visualization.github.io/folium/latest/user\_guide/ui\_elements/icons.html}{Documents}) . In this case, it contains a dictionary with the following settings: \sphinxcode{\sphinxupquote{"icon"}}: This sets the icon for the marker.

\item {} 
\sphinxAtStartPar
\sphinxcode{\sphinxupquote{folium.Icon(...)}}: This specifies the style of the marker icon. Here, it’s configured with a red color \sphinxcode{\sphinxupquote{color="red"}} (“red”) and a star icon {\color{red}\bfseries{}``}icon=”star”{\color{red}\bfseries{}``}(“star”) from the \sphinxhref{https://docs.fontawesome.com/apis/javascript/icon-library}{Font Awesome icon} library {\color{red}\bfseries{}``}prefix=’fa’{\color{red}\bfseries{}``}(“fa”). also you can customize your icon with \sphinxhref{https://icons.getbootstrap.com/}{Bootstrap} icon library.

\end{itemize}

\begin{sphinxVerbatim}[commandchars=\\\{\}]
\PYG{n}{coastalpt\PYGZus{}args} \PYG{o}{=} \PYG{p}{\PYGZob{}}
    \PYG{l+s+s2}{\PYGZdq{}}\PYG{l+s+s2}{m}\PYG{l+s+s2}{\PYGZdq{}}\PYG{p}{:} \PYG{n}{m}\PYG{p}{,} \PYG{c+c1}{\PYGZsh{} specifies the folium map (m)}
    \PYG{l+s+s2}{\PYGZdq{}}\PYG{l+s+s2}{marker\PYGZus{}type}\PYG{l+s+s2}{\PYGZdq{}}\PYG{p}{:} \PYG{l+s+s2}{\PYGZdq{}}\PYG{l+s+s2}{marker}\PYG{l+s+s2}{\PYGZdq{}}\PYG{p}{,} \PYG{c+c1}{\PYGZsh{}specifies the type of marker}
    \PYG{l+s+s2}{\PYGZdq{}}\PYG{l+s+s2}{popup}\PYG{l+s+s2}{\PYGZdq{}}\PYG{p}{:} \PYG{k+kc}{True}\PYG{p}{,} \PYG{c+c1}{\PYGZsh{}Show information as popup when curser move on to the polygon}
    \PYG{l+s+s2}{\PYGZdq{}}\PYG{l+s+s2}{legend}\PYG{l+s+s2}{\PYGZdq{}}\PYG{p}{:} \PYG{k+kc}{False}\PYG{p}{,} \PYG{c+c1}{\PYGZsh{} Don`t display a separated legend.}
    \PYG{l+s+s2}{\PYGZdq{}}\PYG{l+s+s2}{marker\PYGZus{}kwds}\PYG{l+s+s2}{\PYGZdq{}}\PYG{p}{:} \PYG{p}{\PYGZob{}}\PYG{l+s+s2}{\PYGZdq{}}\PYG{l+s+s2}{icon}\PYG{l+s+s2}{\PYGZdq{}}\PYG{p}{:} \PYG{n}{folium}\PYG{o}{.}\PYG{n}{Icon}\PYG{p}{(}\PYG{n}{color}\PYG{o}{=}\PYG{l+s+s2}{\PYGZdq{}}\PYG{l+s+s2}{red}\PYG{l+s+s2}{\PYGZdq{}}\PYG{p}{,} \PYG{n}{icon}\PYG{o}{=}\PYG{l+s+s2}{\PYGZdq{}}\PYG{l+s+s2}{star}\PYG{l+s+s2}{\PYGZdq{}}\PYG{p}{,} \PYG{n}{prefix}\PYG{o}{=}\PYG{l+s+s1}{\PYGZsq{}}\PYG{l+s+s1}{fa}\PYG{l+s+s1}{\PYGZsq{}}\PYG{p}{)}\PYG{p}{\PYGZcb{}} \PYG{c+c1}{\PYGZsh{}style of the marker icon display red color marker with star and refer FontAwesome icon Library}

\PYG{p}{\PYGZcb{}}
\end{sphinxVerbatim}

\sphinxAtStartPar
Display the “coastalpt” Marker on the folium map

\sphinxAtStartPar
The \sphinxcode{\sphinxupquote{.explore()}} visualizes the “coastalpt” GeoDataFrame on the folium map, using the specified parameters.The points are categorized based on the \sphinxstylestrong{“Name”} column, and the marker properties are set according to the \sphinxcode{\sphinxupquote{coastalpt\_args}} dictionary.

\begin{sphinxVerbatim}[commandchars=\\\{\}]
\PYG{c+c1}{\PYGZsh{} Display the \PYGZdq{}coastalpt\PYGZdq{} Marker on the folium map with the customized marker dictionary}
\PYG{n}{coastalpt}\PYG{o}{.}\PYG{n}{explore} \PYG{p}{(}\PYG{l+s+s2}{\PYGZdq{}}\PYG{l+s+s2}{Name}\PYG{l+s+s2}{\PYGZdq{}}\PYG{p}{,} \PYG{o}{*}\PYG{o}{*}\PYG{n}{coastalpt\PYGZus{}args}\PYG{p}{)}
\end{sphinxVerbatim}

\sphinxAtStartPar
Save the created folium map (represented by the m object) as an HTML file.

\sphinxAtStartPar
The \sphinxcode{\sphinxupquote{m.save}} command is used to save the current state of a map (represented by the m object) as an HTML file named “NI\_tourist\_MAP.html”.
You can then open this HTML file in a web browser to view the interactive map.

\begin{sphinxVerbatim}[commandchars=\\\{\}]
\PYG{n}{m}\PYG{o}{.}\PYG{n}{save}\PYG{p}{(}\PYG{l+s+s2}{\PYGZdq{}}\PYG{l+s+s2}{NI\PYGZus{}tourist\PYGZus{}MAP.html}\PYG{l+s+s2}{\PYGZdq{}}\PYG{p}{)}
\end{sphinxVerbatim}

\sphinxAtStartPar
You have successfully generated the tourist map for Northern Ireland.


\section{Troubleshooting}
\label{\detokenize{egm722_serenee/NI_Tourist_Map_doc:troubleshooting}}
\sphinxAtStartPar
If you’re encountering any issues or need assistance with troubleshooting, here are a few steps you can take:
\begin{itemize}
\item {} 
\sphinxAtStartPar
\sphinxstylestrong{Library Imports}: Ensure all the required libraries are installed in your Python environment. If not, you can install them using pip install library\sphinxhyphen{}name.

\item {} 
\sphinxAtStartPar
\sphinxstylestrong{Check File Paths}: Ensure that all file paths provided in the script are correct and that the necessary shapefiles, CSV files, and GeoJSON files are available in the specified locations.

\item {} 
\sphinxAtStartPar
\sphinxstylestrong{Verify Data Loading}: Double\sphinxhyphen{}check that the data loading functions (read\_file, read\_integrated\_csv\_file, read\_tourist\_site\_polygon\_data, read\_geojson\_file, etc.) are correctly reading the data into pandas DataFrames or GeoDataFrames.

\item {} 
\sphinxAtStartPar
\sphinxstylestrong{Inspect Dataframes}: Utilize functions like check\_dataframe\_header, display\_merged\_dataframe\_head, display\_geodataframe\_head, etc., to inspect the loaded dataframes and ensure that they contain the expected data.

\item {} 
\sphinxAtStartPar
\sphinxstylestrong{Dependency Versions}: Ensure that the versions of the Python libraries (pandas, geopandas, folium) used in the code are compatible with each other. Sometimes, certain functionalities might have been deprecated or changed in newer versions of the libraries, leading to unexpected behavior.

\item {} 
\sphinxAtStartPar
\sphinxstylestrong{Coordinate Reference System (CRS)}: Ensure that all your geospatial data is in the same CRS. If not, use the to\_crs method to convert them to a common CRS.

\item {} 
\sphinxAtStartPar
\sphinxstylestrong{Folium Map Display}: If the map is not displaying correctly, ensure that the Jupyter notebook or Python environment you’re using supports inline map display. Sometimes, running the script in a different environment (like JupyterLab or VSCode) can help.

\item {} 
\sphinxAtStartPar
\sphinxstylestrong{Popup Information}: Make sure the column names passed to the popup parameter in the explore method match the actual column names in your GeoDataFrame.

\item {} 
\sphinxAtStartPar
\sphinxstylestrong{Error Messages}: Pay close attention to any error messages you receive when running the script. They often provide valuable clues about what might be going wrong.

\item {} 
\sphinxAtStartPar
\sphinxstylestrong{Saving the Map}: After saving the map as an HTML file, check the file in a web browser to ensure it displays correctly. If it doesn’t, there might be issues with the JavaScript rendering.

\item {} 
\sphinxAtStartPar
\sphinxstylestrong{Debug Functions}: If any custom functions (e.g., create\_visit\_geodataframe, spatial\_join\_geodataframes, etc.) are not producing the desired output, try adding print statements or using a debugger to understand the behavior of the code within those functions.

\item {} 
\sphinxAtStartPar
\sphinxstylestrong{Test Incrementally}: Test each section of your script incrementally to identify where any errors might be occurring. You can comment out sections of the script and run them separately to isolate the problem.

\item {} 
\sphinxAtStartPar
\sphinxstylestrong{Handle Errors}: Ensure that error handling mechanisms are in place, such as try\sphinxhyphen{}except blocks, to catch and handle any exceptions that may arise during execution.

\item {} 
\sphinxAtStartPar
\sphinxstylestrong{Consult Documentation}: Refer to the documentation of libraries like GeoPandas and Folium for guidance on correct usage of functions and methods.

\item {} 
\sphinxAtStartPar
\sphinxstylestrong{Community Support}: If you’re still facing issues, consider reaching out to relevant communities or forums like Stack Overflow, where you can receive assistance from other developers.

\end{itemize}


\section{Reference}
\label{\detokenize{egm722_serenee/NI_Tourist_Map_doc:reference}}\begin{enumerate}
\sphinxsetlistlabels{\arabic}{enumi}{enumii}{}{.}%
\item {} 
\sphinxAtStartPar
M.Breuss (2024), Python Folium:Create Web Maps From Your Data.Real Python. Available at :\sphinxurl{https://realpython.com/python-folium-web-maps-from-data/}

\end{enumerate}

\sphinxAtStartPar
import pdfkit
\begin{description}
\sphinxlineitem{def convert\_html\_to\_pdf(html\_file, pdf\_file):}
\sphinxAtStartPar
pdfkit.from\_file(html\_file, pdf\_file)

\sphinxlineitem{if \_\_name\_\_ == “\_\_main\_\_”:}
\sphinxAtStartPar
convert\_html\_to\_pdf(“NI\_Tourist\_Map\_doc.html”, “test.pdf”)

\end{description}

\sphinxstepscope

\sphinxAtStartPar
\#!/usr/bin/env python
\# coding: utf\sphinxhyphen{}8


\chapter{Intergrated Data Analysis}
\label{\detokenize{egm722_serenee/Integrated_Data_Analysis_doc:intergrated-data-analysis}}\label{\detokenize{egm722_serenee/Integrated_Data_Analysis_doc::doc}}
\sphinxAtStartPar
\# \#\# 1. Fixing Geometry in Shapefiles \#
\#
\# the process of repairing or adjusting the geometric properties of \_\_spatial {\color{red}\bfseries{}data\_\_}, such as \_\_points\_\_, \_\_lines\_\_, or \_\_polygons\_\_, to ensure they meet certain criteria or standards. This could involve tasks such as removing or correcting invalid geometries, simplifying shapes, snapping vertices to a grid, or resolving topological errors. Libraries such as \sphinxtitleref{Shapely} and \sphinxtitleref{GeoPandas} provide functionality to perform these operations efficiently.
\#

\sphinxAtStartPar
\# \#\#\# Getting Started
\# The geopandas package offers a range of functionalities for performing various operations and analyses on vector geospatial data (a {[}GeoDataFrame{]}(\sphinxurl{https://geopandas.org/en/stable/docs/reference/geodataframe.html})).
\#
\# Importing the \sphinxtitleref{geopandas} library with the alias \sphinxtitleref{gpd} ,it provides a shorter and easier\sphinxhyphen{}to\sphinxhyphen{}use name for referencing the library’s functions and objects throughout the code.

\sphinxAtStartPar
\# In{[} {]}:

\sphinxAtStartPar
import geopandas as gpd

\sphinxAtStartPar
\# This code reads a shapefiles located in the data directory using \sphinxtitleref{geopandas} library.
\# Read the country outline shapefile.

\sphinxAtStartPar
\# In{[} {]}:

\sphinxAtStartPar
\# Read the shapefile
input\_data = gpd.read\_file(“data\_files/download\_data/{\color{red}\bfseries{}OSNI\_Open\_Data\_}\sphinxhyphen{}\_Largescale\_Boundaries\_\sphinxhyphen{}\_NI\_Outline.shp”)  \# Path to the input shapefile

\sphinxAtStartPar
\# Fixing the geometry of {[}GeoDataFrame{]}(\sphinxurl{https://geopandas.org/en/stable/docs/reference/geodataframe.html}) by applying the {[}buffer method{]}(\sphinxurl{https://geopandas.org/en/stable/docs/reference/api/geopandas.GeoSeries.buffer.html}) with a Zero distance. This approach can help resolve common geometry issues such as self\sphinxhyphen{}intersections and degenerate geometries.

\sphinxAtStartPar
\# In{[} {]}:

\sphinxAtStartPar
\# Fix geometries using buffer method
fix\_data = input\_data.buffer(0)

\sphinxAtStartPar
\# {[}.to\_crs(){]}(\sphinxurl{https://geopandas.org/en/stable/docs/reference/api/geopandas.GeoDataFrame.to\_crs.html}) method in \sphinxtitleref{GeoPandas} is used to transform all geometries from an active coordinate reference system ({[}CRS{]}(\sphinxurl{https://geopandas.org/en/stable/docs/user\_guide/projections.html})) to another specified CRS. It allows you to convert the spatial data in your \_\_GeoDataFrame\_\_ to a different CRS, ensuring consistency or enabling analysis in a different geographic context.
\#
\# Re\sphinxhyphen{}project the \sphinxtitleref{crs} to WGS84 latitude/longitude {[}(EPSG:4326){]}(https://epsg.io/4326)

\sphinxAtStartPar
\# In{[} {]}:

\sphinxAtStartPar
\# Set CRS to WGS84
fixed\_data = fix\_data.to\_crs(“epsg:4326”)

\sphinxAtStartPar
\# The`.to\_file` {[}(documentation){]}(\sphinxurl{https://geopandas.org/en/stable/docs/reference/api/geopandas.GeoDataFrame.to\_file.html}) function saves the fixed geometries \_\_GeoDataFrame\_\_,into new shapefile.

\sphinxAtStartPar
\# In{[} {]}:

\sphinxAtStartPar
\# Save the fixed geometries to a new shapefile
fixed\_data.to\_file(“data\_files/NI\_Outline.shp”) \# Path to the fixed output shapefile

\sphinxAtStartPar
\# You are free to utilize the output data for analysis.

\sphinxAtStartPar
\# \#\# 2. Coordinate Reference System (CRS) Re\sphinxhyphen{}Projection
\#
\# Coordinate Re\sphinxhyphen{}Projection{[}(documentation){]}(\sphinxurl{https://geopandas.org/en/stable/docs/user\_guide/projections.html\#re-projecting}) is the process of transforming coordinates from one Coordinate Reference System (CRS){[}(documentation){]}(\sphinxurl{https://geopandas.org/en/stable/docs/user\_guide/projections.html}) to another. A \sphinxtitleref{CRS} is a framework used to specify locations on the Earth’s surface. It’s essentially a coordinate\sphinxhyphen{}based system that allows for the precise identification of geographic features and positions.
\#
\# \#\#\# Getting Started
\# Importing the geopandas library with the alias gpd ,it provides a shorter and easier\sphinxhyphen{}to\sphinxhyphen{}use name for referencing the library’s functions and objects throughout the code.

\sphinxAtStartPar
\# In{[} {]}:

\sphinxAtStartPar
import geopandas as gpd

\sphinxAtStartPar
\# This code reads a shapefiles (a {[}\_\_GeoDataFrame\_\_{]}(\sphinxurl{https://geopandas.org/en/stable/docs/reference/geodataframe.html})) located in the data directory using “geopandas” library.

\sphinxAtStartPar
\# In{[} {]}:

\sphinxAtStartPar
\# Read the downloaded shapefiles
tourist\_tmp = gpd.read\_file(“data\_files/download\_data/Historic\_Parks\_and\_Gardens20240410.shp”)  \# Path to the input shapefile of Historic Park and Garden Data.

\sphinxAtStartPar
\# {[}.to\_crs(){]}(\sphinxurl{https://geopandas.org/en/stable/docs/reference/api/geopandas.GeoDataFrame.to\_crs.html}) method in GeoPandas is used to transform all geometries from an active coordinate reference system ({[}CRS{]}(\sphinxurl{https://geopandas.org/en/stable/docs/user\_guide/projections.html})) to another specified CRS. It allows you to convert the spatial data in your GeoDataFrame to a different CRS, ensuring consistency or enabling analysis in a different geographic context.
\#
\# Re\sphinxhyphen{}project the \sphinxtitleref{crs} to WGS84 latitude/longitude {[}(EPSG:4326){]}(https://epsg.io/4326)

\sphinxAtStartPar
\# In{[} {]}:

\sphinxAtStartPar
\# Set CRS to WGS84
tourist\_prj = tourist\_tmp.to\_crs(“epsg:4326”)

\sphinxAtStartPar
\# The`.to\_file` {[}(documentation){]}(\sphinxurl{https://geopandas.org/en/stable/docs/reference/api/geopandas.GeoDataFrame.to\_file.html}) function saves the crs transform GeoDataFrame, into a new shapefile.

\sphinxAtStartPar
\# In{[} {]}:

\sphinxAtStartPar
\# Save the re\sphinxhyphen{}projected shapefile to a new shapefile
tourist\_prj.to\_file(“data\_files/NI\_Tourist\_Sites.shp”) \# Path to the output shapefile

\sphinxAtStartPar
\# You have successfully re\sphinxhyphen{}project the Shapefile.

\sphinxAtStartPar
\# \#\# 3. Clipping shapefiles
\# Clipping shapefiles refers to the process of spatially limiting or cutting down the extent of a shapefile based on the boundary of another shapefile or a defined boundary area. When examining a shapefile of counties in ArcGIS or QGIS, you might notice that county boundaries extend across water features, which can be confusing for map users. To address this, we will refine the map by removing extraneous elements using the country’s outline border.
\#
\# \#\#\# Getting Started
\#
\# Importing the geopandas library with the alias gpd.
\#

\sphinxAtStartPar
\# In{[} {]}:

\sphinxAtStartPar
import geopandas as gpd

\sphinxAtStartPar
\# Read the \_\_GeoDataFrames\_\_ of “counties shapefile” and the \_\_geometry {\color{red}\bfseries{}fixed\_\_} “country outline” data.

\sphinxAtStartPar
\# In{[} {]}:

\sphinxAtStartPar
\# Read the input and mask/clip shapefiles
input\_counties = gpd.read\_file(“data\_files/download\_data/{\color{red}\bfseries{}OSNI\_Open\_Data\_}\sphinxhyphen{}\_Largescale\_Boundaries\_\sphinxhyphen{}\_County\_Boundaries\_.shp”)\# Path to the input shapefile of County Boundaries
clip\_data = gpd.read\_file(“data\_files/NI\_outline.shp”)\# path to the mask shapefile of geometry fixed country boarder

\sphinxAtStartPar
\# To ensure that all files are in a common \sphinxtitleref{CRS} (Coordinate Reference System), we need to reproject the \_\_GeoDataFrames\_\_ using the  {\color{red}\bfseries{}`}.to\_crs`{[}(documentation){]}(\sphinxurl{https://www.google.com/search?q=to\_crs+geopandas\&rlz=1C1JCYX\_jaJP1077JP1077\&oq=.to\_crs+\&gs\_lcrp=EgZjaHJvbWUqBggBEAAYHjIGCAAQRRg5MgYIARAAGB4yBggCEAAYHjIICAMQABgIGB4yCggEEAAYgAQYogQyCggFEAAYgAQYogQyCggGEAAYgAQYogQyCggHEAAYgAQYogTSAQgyODM5ajBqN6gCALACAA\&sourceid=chrome\&ie=UTF-8}) fuction. In the previous step, the outline shapefile has already been reprojected into WGS84. Now, we need to transform the geometries of the GeoDataFrames called “input\_counties” to the desired CRS, which in this case is WGS84.
\#

\sphinxAtStartPar
\# In{[} {]}:

\sphinxAtStartPar
\# Re\sphinxhyphen{}project the CRS of the clipped data (assuming the original data uses the same CRS) WGS84 latitude/longitude(EPSG:4326)
prj\_counties = input\_counties.to\_crs (“epsg:4326”)

\sphinxAtStartPar
\# {\color{red}\bfseries{}`}.crs`{[}(documentation){]}(\sphinxurl{https://geopandas.org/en/latest/docs/reference/api/geopandas.GeoDataFrame.crs.html}) function use for check the existing CRS  of \_\_GeoDataFrames\_\_.

\sphinxAtStartPar
\# In{[} {]}:

\sphinxAtStartPar
Verification of existing CRS
prj\_counties.crs

\sphinxAtStartPar
\# Next, we’re going to clip the counties polygon layer using an outline polygon layer. Both GeoDataFrames have the same Coordinate Reference System (CRS).
\# * \sphinxtitleref{gpd.overlay`{[}(documentation){]}(https://geopandas.org/en/stable/docs/reference/api/geopandas.overlay.html) function overlaying two GeoDataFrames to compute spatial overlay operations, such as intersection, union, or difference.
\#
\# * `prj\_counties} and \sphinxtitleref{clip\_data} represent GeoDataFrames containing geographic data, with prj\_counties likely representing the \_\_input {\color{red}\bfseries{}feature\_\_} and clip\_data representing the \_\_feature used for {\color{red}\bfseries{}clipping\_\_} (e.g., a boundary or mask).
\#
\# * \sphinxtitleref{keep\_geom\_type=True} This parameter specifies whether to keep the geometry types of the input GeoDataFrames in the output GeoDataFrame. By setting it to True, the output will retain the geometry types of both prj\_counties and clip\_data.

\sphinxAtStartPar
\# In{[} {]}:

\sphinxAtStartPar
\#Clipping the counties polygon layer using an outline polygon layer
clipped\_counties = gpd.overlay(prj\_counties, clip\_data, how=’intersection’, keep\_geom\_type=True)

\sphinxAtStartPar
\# Save the result as ashapefile.
\#
\# {\color{red}\bfseries{}`}.to\_fil`{[}(documentation){]}(\sphinxurl{https://geopandas.org/en/stable/docs/reference/api/geopandas.GeoDataFrame.to\_file.html}) function allows to save new GeoDataFrame.

\sphinxAtStartPar
\# In{[} {]}:

\sphinxAtStartPar
\# Save the clipped data to a new shapefile
clipped\_counties.to\_file(“data\_files/NI\_Counties.shp”) \# Path to the fixed output shapefile of County Boundaries

\sphinxAtStartPar
\# You have successfully clipped the counties polygons using country outline data.

\sphinxAtStartPar
\# \#\# 4. Data Intergration
\#
\# Python typically refers to the process of combining data from multiple sources, formats, or databases into a unified format that can be analyzed or used for further processing.
\#
\# \#\#\# i. Integrating GP Surgeries Data by Postal Code
\# We’re going to combine the data from GP surgery with postal code data, creating a unified dataset that includes information from both sources.
\#
\# \#\#\# Getting Started
\# In this step, we will merge two DataFrames and generate a GeoDataFrame as the output.
\#
\# Import the necessary libraries.
\#

\sphinxAtStartPar
\# In{[} {]}:

\sphinxAtStartPar
import pandas as pd
import geopandas as gpd

\sphinxAtStartPar
\# Read the \_\_DataFrame\_\_
\#
\# The \sphinxtitleref{.read\_csv()`{[}(documentation){]}(https://pandas.pydata.org/pandas\sphinxhyphen{}docs/stable/reference/api/pandas.read\_csv.html) function in `Pandas} is used to read data from CSV (Comma Separated Values) files and load it into a \_\_DataFrame\_\_, which is a tabular data structure similar to a spreadsheet or a database table.

\sphinxAtStartPar
\# In{[} {]}:

\sphinxAtStartPar
\# Load csv data sets
uk\_postcodes = pd.read\_csv(“data\_files/download\_data/ukpostcodes.csv”) \# path to UK postal code csv file
gp\_practices = pd.read\_csv(“data\_files/download\_data/gp\sphinxhyphen{}practice\sphinxhyphen{}reference\sphinxhyphen{}file—jan\sphinxhyphen{}2024.csv”) \#path to GP practice csv file

\sphinxAtStartPar
\# Inspect the \_\_DataFrame\_\_ headers to identify a common column that can be used for integrating the \_\_DataFrames\_\_.
\#
\# The \sphinxtitleref{.head()`{[}(documentation){]}(https://pandas.pydata.org/docs/reference/api/pandas.DataFrame.head.html) method in Pandas is used to display the first few rows of a \_\_DataFrame\_\_. By default, it shows the first 5 rows, but you can specify a different number of rows by passing an integer argument to the method `.head(n)}.

\sphinxAtStartPar
\# Check the first few rows of the “uk\_postcodes” DataFrame

\sphinxAtStartPar
\# In{[} {]}:

\sphinxAtStartPar
\# check the header of uk\_postcodes DataFrame
uk\_postcodes.head()

\sphinxAtStartPar
\# Check the first few rows of the “gp\_practices” DataFrame

\sphinxAtStartPar
\# In{[} {]}:

\sphinxAtStartPar
\# check the header of gp\_practices DataFrame
gp\_practices.head()

\sphinxAtStartPar
\# This code merges two DataFrame, namely “uk\_postcode”s and “gp\_practices”, based on a common column in each dataset.
\#
\# * The \sphinxtitleref{.merge`({[}documentation{]}(https://pandas.pydata.org/docs/reference/api/pandas.merge.html)),fuction provided attempts to merge the two datasets. Merge these datasets based on a common field (the “postcode”{[}uk\_postcodes{]} and “Postcode”{[}gp\_practices{]} columns)
\#
\# * The `left\_on}, parameter specifies the column name in the left dataset (uk\_postcodes) to use for merging (in this case, “postcode”).
\#
\# * The \sphinxtitleref{right\_on}, parameter specifies the column name in the right dataset (gp\_practices) to use for merging (in this case, “Postcode”).
\#
\# * \sphinxtitleref{how=”inner”}, parameter specifies the type of merge to perform. In this case, an inner join is performed, meaning only the rows with matching postal codes in both \_\_DataFrames\_\_ will be included in the merged \_\_DataFrame\_\_.

\sphinxAtStartPar
\# In{[} {]}:

\sphinxAtStartPar
\# merge datasets base on postal code
merge\_data = pd.merge(uk\_postcodes, gp\_practices, left\_on=”postcode”, right\_on=”Postcode” , how=”inner”)

\sphinxAtStartPar
\# The \sphinxtitleref{.head()} method is then called on the merged DataFrame to display the first few rows. By default, it shows the first five rows, along with the column names.

\sphinxAtStartPar
\# In{[} {]}:

\sphinxAtStartPar
\#check the header of merged data
merge\_data.head()

\sphinxAtStartPar
\# The downloaded postal code file contains postal codes for the entire United Kingdom (UK). However, for this project, we are only interested in the territory of Northern Ireland, which has postal codes starting with “BT”. In this step, we will filter the rows of the \_\_DataFrame\_\_ to include only those with postal codes starting with “BT”.
\#
\# The \sphinxtitleref{.str.startswith() ` {[}(documentation){]}(https://pandas.pydata.org/docs/reference/api/pandas.Series.str.startswith.html)  method is a string accessor available in `Pandas} that is applied to a Series containing strings. It checks whether each string in the Series starts with a specified prefix and returns a boolean Series indicating the result of this check for each string.
\#
\# * The \sphinxtitleref{.str},is a Pandas accessor that allows you to apply string methods to each element of a Series.
\# * The \sphinxtitleref{.startswith(“BT”)},is a string method that checks if each string in the Series starts with the specified substring, in this case, “BT”. It returns a boolean Series where each element indicates whether the corresponding string starts with “BT”.

\sphinxAtStartPar
\# In{[} {]}:

\sphinxAtStartPar
\# Filtering Northern Ireland postcodes
ni\_postcodes\_tmp = merge\_data{[}merge\_data{[}“postcode”{]}.str.startswith(“BT”){]}

\sphinxAtStartPar
\# In this step, we are removing unnecessary columns from the DataFrame
\#
\# {\color{red}\bfseries{}`}.drop`{[}(documentation){]}(\sphinxurl{https://pandas.pydata.org/pandas-docs/stable/reference/api/pandas.DataFrame.drop.html})  method remove the specified columns on the \_\_DataFrame\_\_. parameter is used to specify the names of the columns to be dropped, provided as a list and returns a new \_\_DataFrame\_\_ with the specified columns removed.

\sphinxAtStartPar
\# In{[} {]}:

\sphinxAtStartPar
\# remove unnecessary columns
ni\_postcodes = ni\_postcodes\_tmp.drop(columns={[}“id” , “Postcode”, “LCG” , “Registered\_Patients”{]})

\sphinxAtStartPar
\# Verify the resulting \_\_DataFrame\_\_ using the \sphinxtitleref{.head} fucntion.

\sphinxAtStartPar
\# In{[} {]}:

\sphinxAtStartPar
\# Verify the resulting DataFrame.
ni\_postcodes.head()

\sphinxAtStartPar
\# We’ll create a new GeoDataFrame from an existing DataFrame.
\#
\# The {\color{red}\bfseries{}`}gpd.GeoDataFrame`{[}(documentation){]}(\sphinxurl{https://geopandas.org/en/stable/docs/reference/api/geopandas.points\_from\_xy.html}) function from the “GeoPandas” library is used to create a \_\_GeoDataFrame\_\_ from a regular DataFrame
\#
\# {\color{red}\bfseries{}`}.points\_from\_xy`{[}(documentation){]}(\sphinxurl{https://geopandas.org/en/stable/docs/reference/api/geopandas.points\_from\_xy.html}) function specifies the geometry column of the GeoDataFrame. In this case, it creates a new geometry column by converting latitude and longitude coordinates into Point geometries. The \_\_longitude\_\_ and \_\_latitude\_\_ columns from the DataFrame “ni\_postcodes” are used as inputs.

\sphinxAtStartPar
\# In{[} {]}:

\sphinxAtStartPar
\# convert the Dataframe to a Geodataframe
ni\_postcodes\_geo = gpd.GeoDataFrame(ni\_postcodes, geometry=gpd.points\_from\_xy(ni\_postcodes.longitude, ni\_postcodes.latitude))

\sphinxAtStartPar
\# Save the GeoDataFrame as a GeoJSON file.
\#
\# * The \sphinxtitleref{.to\_file} {[}(documentation){]}(\sphinxurl{https://geopandas.org/en/stable/docs/reference/api/geopandas.GeoDataFrame.to\_json.html}) method is used to save the GeoDataFrame to a file.
\#
\# * The \sphinxtitleref{driver=”GeoJSON”} parameter specifies the file format to use for saving the data. In this case, we’re using the GeoJSON format.

\sphinxAtStartPar
\# In{[} {]}:

\sphinxAtStartPar
\# Save the filtered dataset
ni\_postcodes\_geo.to\_file(“data\_files/NI\_PostCodes\_GP.geojson”, driver=”GeoJSON”)

\sphinxAtStartPar
\# You have successfully exported the GeoJSON file combining GP surgery details with postal codes.

\sphinxAtStartPar
\# \#\#\# ii. Distance Calculation.
\# We’ll locate the closest transport hub along with its distance and also identify the nearest GP surgery and its distance from the tourist sites.
\#
\# \#\#\# Getting Started
\# In this step, we will handle \_\_GeoDataFrames\_\_ and calculate the distances.
\#
\# Import the necessary libraries.

\sphinxAtStartPar
\# In{[} {]}:

\sphinxAtStartPar
import pandas as pd
import geopandas as gpd

\sphinxAtStartPar
\# Read previously projected tourist sites data(a \_\_GeoDataFrame\_\_).

\sphinxAtStartPar
\# In{[} {]}:

\sphinxAtStartPar
\# Read projected tourist data
tourist = gpd.read\_file(“data\_files/NI\_Tourist\_Sites.shp”)  \# Path to the input shapefile of projected tourist sites

\sphinxAtStartPar
\# Check the first few rows and try to gain a better understanding of its contents.
\# In this dataset, the \sphinxtitleref{geometry} column represents spatial data in the form of “polygons”, which are geometric shapes defined by a series of connected points in space.

\sphinxAtStartPar
\# In{[} {]}:

\sphinxAtStartPar
\# verify the header
tourist.head()

\sphinxAtStartPar
\# Read the GeoJSON files and reproject them into the “Cartesian 2D” coordinate reference system({[}EPSG:2157,IRENET95 / Irish Transverse Mercator{]}(https://epsg.io/2157)).
\#
\# The choice of coordinate system significantly influences the accuracy of \_\_distance {\color{red}\bfseries{}calculations\_\_} {[}(documentation){]}(\sphinxurl{https://automating-gis-processes.github.io/CSC/notebooks/L2/projections.html\#Calculating-distances).This} “Cartesian 2D” coordinate reference system ({[}Projected Coordinate System{]}(\sphinxurl{https://www.esri.com/arcgis-blog/products/arcgis-pro/mapping/gcs\_vs\_pcs/\#:~:text=A\%20projected\%20coordinate\%20system\%20(PCS)\%20is\%20a\%20GCS\%20that\%20has,to\%20know\%20how\%20to\%20draw}.)) is crucial for accurate distance calculations, as it uses “Cartesian coordinates” in “meters”, providing precise measurements over \_\_flat {\color{red}\bfseries{}surfaces\_\_}. WGS84, on the other hand, utilizes longitudes and latitudes in the {[}geographic coordinate system {]}(\sphinxurl{https://desktop.arcgis.com/en/arcmap/latest/map/projections/about-geographic-coordinate-systems.htm\#:~:text=A\%20geographic\%20coordinate\%20system\%20(GCS,(based\%20on\%20a\%20spheroid}).) , which may not yield accurate distance measurements due to the Earth’s curved surface.
\#

\sphinxAtStartPar
\# Read the downloaded Transport hub “geojson” data (a \_\_GeoDataFrame\_\_)

\sphinxAtStartPar
\# In{[} {]}:

\sphinxAtStartPar
\#Read the downloaded Transport hub “geojson” and transform to Irish Transverse Mercator
transport = gpd.read\_file(“data\_files/download\_data/translink\sphinxhyphen{}stations\sphinxhyphen{}ni.geojson”).to\_crs(“epsg:2157”)

\sphinxAtStartPar
\# Read the previously integrated “geojson” dataset containing GP surgeries and postal codes (a \_\_GeoDataFrame\_\_).

\sphinxAtStartPar
\# In{[} {]}:

\sphinxAtStartPar
\#Read the previously integrated “geojson” dataset containing GP surgeries and postal codes
\# and re\sphinxhyphen{}Projected to Irish Transverse Mercator coordinate reference system
post\_gp = gpd.read\_file(“data\_files/NI\_PostCodes\_GP.geojson”).to\_crs(“epsg:2157”)

\sphinxAtStartPar
\# This code snippet iterates over each tourist site in the GeoDataFrame tourist, calculates the distance to the nearest bus/train station and the nearest GP surgery, and records the shortest distance in kilometers along with the name of the station.
\#
\# explanation of the code:
\#
\# * The \sphinxtitleref{for ind, row in tourist.to\_crs(“epsg:2157”).iterrows()} parameters “iterates over”(\sphinxtitleref{.iterrows()`{[}documentation{]}(https://pandas.pydata.org/docs/reference/api/pandas.DataFrame.iterrows.html)) each row (“tourist”) in the \_GeoDataFrame\_ tourist, after projecting it to the Irish Transverse Mercator coordinate reference system (EPSG:2157) for distance calculations in meters.
\#
\# * The `pt = row{[}“geometry”{]}.centroid} {[}(documentation){]}(\sphinxurl{https://geopandas.org/en/stable/docs/reference/api/geopandas.GeoSeries.centroid.html}): function calculates the centroid of each tourist site geometry(polygon).
\#
\# * The \sphinxtitleref{.distance()`{[}(documentation){]}(https://geopandas.org/en/stable/docs/reference/api/geopandas.GeoSeries.distance.html) function calculates the distance between the centroid of the tourist site and all train/bus stations stored in the \_\_GeoDataFrame\_\_, as well as all GP practices stored in the \_\_GeoDataFrame\_\_.
\#
\# * The}.argmin()`{[}(documentation){]}(\sphinxurl{https://pandas.pydata.org/docs/reference/api/pandas.Series.argmin.html}) method retrieves the index of the train/bus station with the shortest distance to the tourist site, as well as the index of the nearest GP surgery.
\#
\# * The \sphinxtitleref{.min()`{[}(documentation){]}(https://docs.python.org/3/library/functions.html\#min) method retrieves the shortest distance to the nearest train/bus station in meters, as well as the shortest distance to the nearest GP surgery.
\# * The `.loc`{[}(documentation){]}(https://pandas.pydata.org/docs/reference/api/pandas.DataFrame.loc.html) indexer is a `Pandas} method used for label\sphinxhyphen{}based indexing. It is primarily used to access and modify specific rows and columns of a \_\_DataFrame\_\_ based on their labels (row and column names). It allows you to select data by specifying the row labels and column names explicitly.
\# * The \sphinxtitleref{tourist.loc{[}ind, “Near\_T\_Hub”{]} = transport.loc{[}min\_ind\_trans{]}.Station.title()} , line assigns the name of the nearest transport hub (train/bus station) to the “Near\_T\_Hub” column for the current tourist site. The \sphinxtitleref{.title()`{[}(documentation){]}(https://docs.python.org/3/library/stdtypes.html\#str.title) is a built\sphinxhyphen{}in Python string method it capitalizes the first letter of each word in the station name.
\# Omit the use of the `.title()} method when assigning to the “GP practice name” because preserving the original case of the names is necessary for combining data in the next steps.The \sphinxtitleref{.merge} function is Case sensitive.
\#
\# * \sphinxtitleref{/1000} function convert meters into kilometers.
\#

\sphinxAtStartPar
\# In{[} {]}:

\sphinxAtStartPar
\# for each tourist site centroid,find the closest bus/train station
\# Record the Shortest distance in km and the name of the station.
for ind, row in tourist.to\_crs(“epsg:2157”).iterrows():
\begin{quote}

\sphinxAtStartPar
pt = row{[}“geometry”{]}.centroid \# get the centroid of the each tourist site polygon.

\sphinxAtStartPar
distance\_trans = transport.distance(pt) \# find the distance between the centroid and all train station
distance\_postgp = post\_gp.distance(pt) \# find the distance between the centroid and all GP pratices

\sphinxAtStartPar
min\_ind\_trans = distance\_trans.argmin() \# get the index of minimum value
min\_dist\_trans = distance\_trans.min() \# get the minimum distance

\sphinxAtStartPar
min\_ind\_postgp = distance\_postgp.argmin()
min\_dist\_postgp = distance\_postgp.min()

\sphinxAtStartPar
\# define column header
tourist.loc{[}ind, “Near\_T\_Hub”{]} = transport.loc{[}min\_ind\_trans{]}.Station.title() \# assigns the name of the nearest transport hub and capitalizes the first letter of each word in the station name
tourist.loc{[}ind, “Near\_GP”{]} = post\_gp.loc{[}min\_ind\_postgp{]}.PracticeName \#assigns the name of the nearest GP Surgery

\sphinxAtStartPar
\# add distance to the closest transport hub
tourist.loc{[}ind, “Trans\_Dist”{]} = min\_dist\_trans / 1000 \# distance in km
tourist.loc{[}ind, “GP\_Dist”{]} = min\_dist\_postgp / 1000
\end{quote}

\sphinxAtStartPar
\# The {\color{red}\bfseries{}`}.round()`{[}(documentation){]}(\sphinxurl{https://pandas.pydata.org/docs/reference/api/pandas.DataFrame.round.html})  is used to round the distances to a specified number of decimal places.In this case, the distance values are being rounded to centimeter accuracy.

\sphinxAtStartPar
\# In{[} {]}:

\sphinxAtStartPar
\# round the distance to cm accuracy (2 decimal places)
tourist.Trans\_Dist = tourist.Trans\_Dist.round(2)
tourist.GP\_Dist = tourist.GP\_Dist.round(2)

\sphinxAtStartPar
\# First, let’s check the first few rows of the resulting “tourist” \_\_GeoDataFrame\_\_. Then, we’ll inspect the “Near\_GP” column and verify that all row values are in uppercase. If any row values are not uppercase, we’ll need to go back to the previous cell and ensure that the \sphinxtitleref{.title()} function has been removed for the “post\_gp” \_\_GeoDataFrame\_\_.

\sphinxAtStartPar
\# In{[} {]}:

\sphinxAtStartPar
\#check the header and verify that all index in the “PracticeName” column are in uppercase.
tourist.head() \# check the header of final data set

\sphinxAtStartPar
\# Inspect the first few rows of the post\_gp GeoDataFrame and examine the “PracticeName” column. You’ll observe that all row values are in uppercase.

\sphinxAtStartPar
\# In{[} {]}:

\sphinxAtStartPar
\#check the header
post\_gp.head()

\sphinxAtStartPar
\#
\# You’ll notice that both columns have common row values with uppercase letters. Maintaining case sensitivity is crucial when merging \_\_DataFrames\_\_.

\sphinxAtStartPar
\# Filter the necessary columns from the “tourist” \_\_GeoDataFrame\_\_ and create a new \_\_DataFrame\_\_ named “tourist\_out”.

\sphinxAtStartPar
\# In{[} {]}:

\sphinxAtStartPar
\# Filter necessary columns
tourist\_out = pd.DataFrame(tourist{[}{[}“SITE”, “Near\_T\_Hub”,”Trans\_Dist”,”Near\_GP”, “GP\_Dist”{]}{]})

\sphinxAtStartPar
\# Inspect the first few rows of the tourist\_out DataFrame

\sphinxAtStartPar
\# In{[} {]}:

\sphinxAtStartPar
\# verify the header
tourist\_out.head()

\sphinxAtStartPar
\# Merge the \_\_DataFrame\_\_ “post\_gp” with the \_\_GeoDataFrame\_\_ “tourist\_out” based on a common column, which is “PracticeName” from “post\_gp” and “Near\_GP” from “tourist\_out”. The \sphinxtitleref{.merge} operation is performed using an \_\_inner {\color{red}\bfseries{}join\_\_}, ensuring that only rows with matching values in both DataFrames are included in the resulting merged DataFrame named “merge”.
\#
\# * \sphinxtitleref{pd.merge()}: This function from the Pandas library is used to merge two \_\_DataFrames\_\_.
\# * \sphinxtitleref{post\_gp} and \sphinxtitleref{tourist\_out}: These are the \_\_DataFrames\_\_ to be merged.
\# * \sphinxtitleref{left\_on=”PracticeName”} and \sphinxtitleref{right\_on=”Near\_GP”}: These parameters specify the columns from the left and right \_\_DataFrames\_\_ that will be used for the merge operation. In this case, “PracticeName” from post\_gp and “Near\_GP” from tourist\_out are used.
\# * \sphinxtitleref{how=”inner”}: This parameter specifies the type of merge to perform. In this case, an inner join is performed, meaning only the rows with matching values in both \_\_DataFrames\_\_ will be included in the merged DataFrame.

\sphinxAtStartPar
\# In{[} {]}:

\sphinxAtStartPar
\# Merge DataFrame and GeoDataFrame
merged = pd.merge(post\_gp, tourist\_out,left\_on=”PracticeName”, right\_on=”Near\_GP”, how=”inner”)

\sphinxAtStartPar
\# Inspect the first few rows of the “merged” \_\_DataFrame\_\_

\sphinxAtStartPar
\# In{[} {]}:

\sphinxAtStartPar
\#verify the header
merged.head()

\sphinxAtStartPar
\# This code filters the necessary columns from the merged \_\_DataFrame\_\_ “merged”, creates a new \_\_DataFrame\_\_ named output, renames the columns, and saves the result to a CSV file.
\# * The \sphinxtitleref{pd.DataFrame(…)`{[}(documentation){]}(https://pandas.pydata.org/docs/reference/api/pandas.DataFrame.html) function creates a new \_\_DataFrame\_\_ named output with the filtered columns from the previous step.
\# * The `.rename`{[}(documentation){]}(https://pandas.pydata.org/docs/reference/api/pandas.DataFrame.rename.html) function  is used to rename specified columns to defined new names.
\# * The `inplace=True} function ensures that the changes are made directly to the output DataFrame.
\# * The {\color{red}\bfseries{}`}.to\_csv`{[}(documentation){]}(\sphinxurl{https://pandas.pydata.org/pandas-docs/stable/reference/api/pandas.DataFrame.to\_csv.html}) function saves the output DataFrame to a CSV file.

\sphinxAtStartPar
\# In{[} {]}:

\sphinxAtStartPar
\# This code filters the necessary columns
\#rename specified columns to defined new names
\#saves the output DataFrame to a CSV file.
output = pd.DataFrame(merged{[}{[}“SITE”, “Near\_T\_Hub”,”Trans\_Dist”,”Near\_GP”, “GP\_Dist”,”postcode”{]}{]})
output.rename(columns=\{“SITE”:”Tourist Sites”, “postcode”:”PostCode”\},inplace=True)
output.to\_csv(“data\_files/NI\_Tourist\_trans\_GP\_Dist.csv”)

\sphinxAtStartPar
\# \#\#\# iii. Coastline spots intergration
\#
\# We’ll locate the closest transport hub along with its distance and also identify the nearest GP surgery and its distance from the Coastline spots.
\#
\# \#\#\# Getting Started
\#
\# The necessary libraries have already been imported in section 4. If you’d like to start from the middle, navigate back to section 4 and import the libraries. Alternatively, you can define new libraries by adding a new cell here.

\sphinxAtStartPar
\# Read the downloaded shapefile data (a \_\_GeoDataFrame\_\_) in coastline.

\sphinxAtStartPar
\# In{[} {]}:

\sphinxAtStartPar
\# Read Places\_to\_Visit\_in\_Causeway\_Coast\_and\_Glens
coastline\_tmp = gpd.read\_file(“data\_files/download\_data/Places\_to\_Visit\_in\_Causeway\_Coast\_and\_Glens.shp”) \# Path to the input shapefile of Places to Visit in coastaline data

\sphinxAtStartPar
\# Check few rows using \sphinxtitleref{.head()} function

\sphinxAtStartPar
\# In{[} {]}:

\sphinxAtStartPar
\# check the header
coastline\_tmp.head()

\sphinxAtStartPar
\# The code functions similarly to the previous step, with the main difference being that the “coastal\_tmp” geometry represents points, so there’s no need to calculate the centroid as done previously.

\sphinxAtStartPar
\# In{[} {]}:

\sphinxAtStartPar
\# for each coastline spots ,find the closest bus/train station
\# Record the Shortest distance in km and the name of the station.
for ind, row in coastline\_tmp.to\_crs(“epsg:2157”).iterrows():
\begin{quote}

\sphinxAtStartPar
pt = row{[}“geometry”{]} \# consider geometry point.

\sphinxAtStartPar
distance\_trans = transport.distance(pt) \# find the distance between the coastline spots and all train station
distance\_postgp = post\_gp.distance(pt) \# find the distance between the coastline spots and all GP pratices

\sphinxAtStartPar
min\_ind\_trans = distance\_trans.argmin() \# get the index of minimum value
min\_dist\_trans = distance\_trans.min() \# get the minimum distance

\sphinxAtStartPar
min\_ind\_postgp = distance\_postgp.argmin()
min\_dist\_postgp = distance\_postgp.min()

\sphinxAtStartPar
\# define column header
coastline\_tmp.loc{[}ind, “Near\_T\_Hub”{]} = transport.loc{[}min\_ind\_trans{]}.Station.title() \# assigns the name of the nearest transport hub and capitalizes the first letter of each word in the station name
coastline\_tmp.loc{[}ind, “Near\_GP”{]} = post\_gp.loc{[}min\_ind\_postgp{]}.PracticeName \#assigns the name of the nearest GP Surgery

\sphinxAtStartPar
\# add distance to the closest transport hub
coastline\_tmp.loc{[}ind, “Trans\_Dist”{]} = min\_dist\_trans / 1000 \# distance in km
coastline\_tmp.loc{[}ind, “GP\_Dist”{]} = min\_dist\_postgp / 1000
\end{quote}

\sphinxAtStartPar
\# Distance rounding to the centimeter accuracy.

\sphinxAtStartPar
\# In{[} {]}:

\sphinxAtStartPar
\# round the distance to cm accuracy (2 decimal places)
coastline\_tmp.Trans\_Dist = coastline\_tmp.Trans\_Dist.round(2)
coastline\_tmp.GP\_Dist = coastline\_tmp.GP\_Dist.round(2)

\sphinxAtStartPar
\# Verify the results

\sphinxAtStartPar
\# In{[} {]}:

\sphinxAtStartPar
\#verify the header
coastline\_tmp.head()

\sphinxAtStartPar
\# This code creates a GeoDataFrame containing selected columns. It then converts the CRS to WGS84 and saves as a GeoJSON file.

\sphinxAtStartPar
\# In{[} {]}:

\sphinxAtStartPar
\# This code filters the necessary columns
\# rename specified columns to defined new names
\# saves the output DataFrame to a CSV file.
coastal\_out = gpd.GeoDataFrame(coastline\_tmp{[}{[}“Name”, “Website”,”geometry”, “Near\_T\_Hub”,”Trans\_Dist”,”Near\_GP”, “GP\_Dist”,”Postcode”{]}{]})
coastal\_out.to\_crs(“epsg:4326”)
coastal\_out.to\_file(“data\_files/NI\_Coastal\_spots.geojson”,driver=”GeoJSON”)

\sphinxAtStartPar
\# In{[} {]}:

\sphinxstepscope


\chapter{Sphinx and HTML Web Publishing}
\label{\detokenize{Other/Sphinx_and_HTML_Web_Publishing:sphinx-and-html-web-publishing}}\label{\detokenize{Other/Sphinx_and_HTML_Web_Publishing::doc}}

\section{Introduction}
\label{\detokenize{Other/Sphinx_and_HTML_Web_Publishing:introduction}}
\sphinxAtStartPar
This guide will detail the steps for setting up \sphinxstylestrong{Sphinx using pip}, \sphinxstylestrong{creating documentation}, and \sphinxstylestrong{publishing HTML} file on GitHub Pages.


\section{What is sphinx}
\label{\detokenize{Other/Sphinx_and_HTML_Web_Publishing:what-is-sphinx}}
\sphinxAtStartPar
\sphinxhref{https://www.sphinx-doc.org/en/master/}{Sphinx} is a documentation generation tool that allows you to create high\sphinxhyphen{}quality documentation for Python projects and other programming languages.
It enables you to write documentation in reStructuredText or Markdown formats and then generates various output formats such as HTML, PDF, ePub, and more. Sphinx is widely used in the software development community to create well\sphinxhyphen{}structured and easily navigable documentation for projects.
Additionally, Sphinx can parse and render NumPy docstrings, making it an extension of the NumPy docstring format and thus particularly suited for documenting Python libraries and scientific computing projects.


\section{Doc String format}
\label{\detokenize{Other/Sphinx_and_HTML_Web_Publishing:doc-string-format}}
\sphinxAtStartPar
\sphinxcode{\sphinxupquote{NumPy docstring}} and \sphinxcode{\sphinxupquote{Google docstring}} are two major types of docstrings commonly used in Python programming, particularly in the context of writing documentation for functions, classes, and modules.

\sphinxAtStartPar
\sphinxstylestrong{NumPy Docstring Style:} (\sphinxhref{https://numpydoc.readthedocs.io/en/latest/format.html\#overview}{Documentation})
This style is often used in the scientific computing and data analysis community, especially when documenting functions and classes related to \sphinxcode{\sphinxupquote{NumPy}} and \sphinxcode{\sphinxupquote{SciPy}} libraries.
NumPy docstrings typically follow a specific format, including sections such as Parameters, Returns, Examples, and Notes.
This format aims to provide detailed information about the purpose, inputs, outputs, and usage examples of the documented function or class.

\sphinxAtStartPar
\sphinxstylestrong{Google Docstring Style:} (\sphinxhref{https://google.github.io/styleguide/pyguide.html}{Documentation})
The Google docstring style is a more general\sphinxhyphen{}purpose style used for documenting Python code across various projects.
It is based on conventions developed within Google and has gained popularity due to its simplicity and readability.
Google docstrings typically include a one\sphinxhyphen{}line summary, followed by a more detailed description, and sections like Args, Returns, Raises, Examples, etc.
This style doesn’t mandate any specific markup language but is often written in plain text or reStructuredText.
Google docstrings focus on providing clear and concise documentation that is easy to read and understand.


\section{What is pip}
\label{\detokenize{Other/Sphinx_and_HTML_Web_Publishing:what-is-pip}}
\sphinxAtStartPar
\sphinxhref{https://pypi.org/project/pip/}{pip} is the package installer for Python. It’s a command\sphinxhyphen{}line tool that allows you to install, upgrade, and manage Python packages from the Python Package Index (PyPI) and other package indexes.
With pip, you can easily install libraries and frameworks developed by the Python community, enabling you to extend Python’s functionality for various tasks such as web development, data analysis, machine learning, and more.


\section{Getting Started}
\label{\detokenize{Other/Sphinx_and_HTML_Web_Publishing:getting-started}}

\section{Setting up GitHub}
\label{\detokenize{Other/Sphinx_and_HTML_Web_Publishing:setting-up-github}}

\section{Install pip}
\label{\detokenize{Other/Sphinx_and_HTML_Web_Publishing:install-pip}}
\sphinxAtStartPar
\sphinxstylestrong{Using cmd}
First, ensure you have Python installed on your system. pip comes bundled with Python by default from Python 3.4 onwards.
You can check if Python is installed by running \sphinxcode{\sphinxupquote{python \sphinxhyphen{}\sphinxhyphen{}version}} or \sphinxcode{\sphinxupquote{python3 \sphinxhyphen{}\sphinxhyphen{}version}} in your terminal or command prompt.
If Python is installed, you can check if pip is already installed by running \sphinxcode{\sphinxupquote{pip \sphinxhyphen{}\sphinxhyphen{}version}} or \sphinxcode{\sphinxupquote{pip3 \sphinxhyphen{}\sphinxhyphen{}version}} in your terminal or command prompt.
If pip is not installed or you have an older version, you can install or upgrade it using the appropriate command based on your Python version:

\sphinxAtStartPar
For Python 2.x:

\sphinxAtStartPar
For Python 3.x:

\sphinxAtStartPar
If you’re using macOS or Windows, you can download the appropriate Python installer from the \sphinxhref{https://www.python.org/downloads/}{official Python website} which should come with pip.
After installation, you can verify that pip is installed correctly by running \sphinxcode{\sphinxupquote{pip \sphinxhyphen{}\sphinxhyphen{}version}} or \sphinxcode{\sphinxupquote{pip3 \sphinxhyphen{}\sphinxhyphen{}version}} again.

\sphinxAtStartPar
** Using \sphinxcode{\sphinxupquote{Conda}} package manager **
\begin{itemize}
\item {} 
\sphinxAtStartPar
Activate your Conda environment: Before installing packages with pip, it’s recommended to activate your Conda environment to ensure that the packages are installed within the environment.

\end{itemize}
\begin{itemize}
\item {} 
\sphinxAtStartPar
Install pip into your Conda environment (if not already installed): In most cases, Conda environments come with pip pre\sphinxhyphen{}installed. However, if it’s not installed or you want to ensure you have the latest version, you can install it using Conda.

\end{itemize}

\sphinxAtStartPar
or with \sphinxcode{\sphinxupquote{conda}} via conda\sphinxhyphen{}forge:

\begin{sphinxVerbatim}[commandchars=\\\{\}]

\end{sphinxVerbatim}

\sphinxAtStartPar
conda install \sphinxhyphen{}c conda\sphinxhyphen{}forge pip

\sphinxAtStartPar
(\sphinxcode{\sphinxupquote{\sphinxhyphen{}c conda\sphinxhyphen{}forge}}: This specifies the channel from which to install the package. Channels are locations where packages are stored and can be accessed. \sphinxcode{\sphinxupquote{conda\sphinxhyphen{}forge}} is a popular community channel for Conda packages maintained by the community.)
\begin{itemize}
\item {} 
\sphinxAtStartPar
Use pip to install the desired packages: Once pip is installed within your Conda environment, you can use it just like you would in a regular Python environment.

\end{itemize}

\sphinxAtStartPar
Explore available packages on \sphinxhref{https://pypi.org/}{PyPI}


\section{Install sphinx}
\label{\detokenize{Other/Sphinx_and_HTML_Web_Publishing:install-sphinx}}
\sphinxAtStartPar
To install Sphinx using pip, you can use the following command:

\sphinxAtStartPar
or with \sphinxcode{\sphinxupquote{conda}} via conda\sphinxhyphen{}forge:

\begin{sphinxVerbatim}[commandchars=\\\{\}]

\end{sphinxVerbatim}

\sphinxAtStartPar
conda install \sphinxhyphen{}c conda\sphinxhyphen{}forge sphinx is this corrrect


\chapter{Indices and tables}
\label{\detokenize{index:indices-and-tables}}\begin{itemize}
\item {} 
\sphinxAtStartPar
\DUrole{xref,std,std-ref}{genindex}

\item {} 
\sphinxAtStartPar
\DUrole{xref,std,std-ref}{modindex}

\item {} 
\sphinxAtStartPar
\DUrole{xref,std,std-ref}{search}

\end{itemize}



\renewcommand{\indexname}{Index}
\printindex
\end{document}